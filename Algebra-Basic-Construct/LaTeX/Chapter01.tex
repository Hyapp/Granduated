\chapter{集合论}


\begin{itemize}
    \item Zermelo-Fraenkel公理集合论和选择公理的明确表述
    \item 序数理论和Zorn引理 
    \item 关于基数的基本定义和操作
    \item 阐述Grothendieck宇宙的概念,借以安全地运用范畴论 
\end{itemize}

\section{ZFC公理一览}
    Zermelo-Fraenkel公理集合论与选择公理仅作粗略勾画。
\begin{itemize}
    \item 一阶逻辑谓词 $\vee, \wedge, \lnot, \forall ,\exists, \Rightarrow $与量词,括号$()$,变元$x,y,z ...$和等号$=$(视为二元谓词),合规者称为合式公式
    \item 二元谓词$\in$和下列9条公理构成了公理集合论的基础。
\end{itemize}
    ZFC处理的所有对象$x,y$都应该理解为集合;特别的,集合的元素也可以是集合。集合$s$和仅含$s$的集合$\{s\}$是不同的集合。

    公理集合论九公理:
    \begin{itemize}
        \item 外延公理: $\forall x \in A \rightarrow  x \in B \wedge \forall y \in B\rightarrow  y \in A \Rightarrow A=B$
        $\forall X,\emptyset =\{u\in X : u \neq u\}$ 此时空集与集合$X$的选取无关,前提是集合确实存在,可以视为"存在公理"或者由 无穷公理 保证。
        \item 配对公理: $\forall x,y\rightarrow \exists \{x,y\}$
        可以据此定义有序对的概念:$(x,y):=\{\{x\},\{x,y\}\}$
        \item 分离公理模式: 设$P$为关于集合的一个性质,$P(u)$表示$u$满足$P$,则对任意集合$X$存在集合$Y=\{u\in X: P(u)\}$
        \item 并集公理: $\forall X, \exists \bigcup X:=\{u:\exists v\in X\rightarrow u\in v\}$.
        公理中的$X$通常视为集族。 $\bigcup X=\bigcup_{v\in X} v$
        \item 幂集公理: 任意集合$X$,其子集构成集合$P(X):=\{u:u \subset X\}$
        \item 无穷公理: 存在无穷集
        无穷集可以用形式语言表示:$\exists x[(\emptyset \in x) \wedge \forall y\in x(y\bigcup\{y\}\in x)]$.

        $x$称为归纳集,归纳集用于萃取其子集
        $$
        \{
            \emptyset,
            \{
                \emptyset
            \},
            \{
                \emptyset,
                \{
                    \emptyset 
                \}
            \}
            ,
            ... 
        \}
        $$
        \item 替换公理模式: $F$是以集合$X$为定义域的函数$\rightarrow \exists F(X)=\{F(x):x\in X\}$
        
        这里的函数F和分离公理模式中的性质P并非函数,它们是由合式$\forall x\exists !y \varphi(x,y)$. 对每个$P \vee F$产生一条公理,因此称为公理模式。
        \item 正则公理: 任何非空集合都含有对从属关系$\in$的极小元。
        
        正则公理的重要推论是$\forall x,\lnot$无穷链$(x \ni x_1 \ni x_2 \ni ...)$,特别的$x \notin x$

        \item 选择公理: 集合$X$的每个元素非空,$\exists g:X\rightarrow \bigcup X$使得$\forall x \in X, g(x)\in x$称$g$为选择函数。\label{Axiom of Choose}
        
        这里的选择函数$g$从每个$x$中挑选了一个元素
    \end{itemize}
    严格来说,在ZFC的形式系统内仅能谈论集合。其它集合论系统(NBG(von Neumann-Bernays-Godel))还定义了类,凡是集合都是类,凡不是集合的类称为\space 真类,如所有的集合。
\begin{Def}[等价关系]
    满足三性质的(二元)关系$~$称为等价关系
    \begin{enumerate}
        \item 反身性 $x \thicksim x$
        \item 对称性 $x \thicksim y\Rightarrow y \thicksim x$
        \item 传递性 $(x \thicksim y)\wedge (y\thicksim z)\Rightarrow x\thicksim z$ 
    \end{enumerate}
\end{Def}
在ZFC内可以定义一集空集的并,但无法定义空集的交,这需要使用全体集合的类$V$,超出了ZFC的范畴。

\section{序结构与序数}
按照Bourbaki的观点,序结构、拓扑结构、代数结构组成了数学结构的三大母体。
\begin{Def}[偏序关系]
    $(P,\leq)$,$P$是集合,$\leq $是$P$上的二元关系,满足
    \begin{enumerate}
        \item 反身性 $x\leq x$
        \item 传递性 $x\leq y \wedge y\leq z\Rightarrow x\leq z$
        \item 反称性 $x\leq y \wedge y \leq x\Rightarrow x=y$
    \end{enumerate}
    称$\leq $为$P$上的偏序关系
\end{Def}

\Def[预序集]{仅满足反身性与传递性}
\Def[保序映射]{预序集间满足$x\leq y\Rightarrow f(x)\leq f(y)$的映射}
\Def[序型]{偏序集的同构类}
\begin{Def}[极大元,上界,上确界]
    $P' \subset P,P$是预序集
    \begin{itemize}
        \item 称$x\in P$是$P$的极大元, If 不存在$x<y$的元素$y. y\in P$
        \item 称$x\in P$是$P'$的上界, If $\forall x' \in P', x' \leq x$
        \item 称$x\in P$是$P'$的上确界, If $x$是$P'$的上界 $\wedge \forall $上界$y,x\leq y$
    \end{itemize}    
    根据反称性,上(下)确界若存在则惟一???.记为$\sup{P'},\inf{P'}$
\end{Def}

\begin{Def}[滤过序集]
    偏序集$P \neq \emptyset, \forall x,y \in P,\{x,y\}$都有上界,称$P$是滤过序集
\end{Def}

\begin{Def}[全序集(线序集、链)]
    $\forall x,y \in P\Rightarrow x\leq y \vee y \leq x$ 称$P$为全序集(线序集、链)
\end{Def}

有时候也可以讨论类上的序或全序、上下界等定义
\begin{Def}[良序集(G. Cantor)]
    每个$P$的非空子集都有极小元
\end{Def}

\begin{Exap}[滤过序集、全序集、良序集的例子] 
    \begin{itemize}
        \item \space
        \item 任意集合$X, (P(X),\subset)$是滤过偏序集,但一般不是全序
        \item 非负整数集$\mathbb{Z}_{\geq 0}$ 和 $\mathbb{R}$都有标准的全序结构
        \item $\mathbb{Z}_{\geq 0}$是良序集,$\mathbb{R}$则不是
    \end{itemize}
\end{Exap}

\begin{Lem}[良序集上的严格增算子必增]
    设$P$为良序集,映射$\phi :P\rightarrow P$严格增,则$\forall x \in P \Rightarrow \phi (x)\geq x$ 
    
    特别地:
    \begin{enumerate}
        \item $P$没有非平凡的自同构
        \item $\forall x\in P$, 不存在$P\rightarrow P_{<x}:=\{y\in P: y<x\}$的同构   
    \end{enumerate}
    \begin{proof}
        略
    \end{proof}
\end{Lem}

$P_{<x}$的子集继承$P$的良序,称作是$P$的一个前段

序数的想法起源是良序集的序型;采用此方法则必须考虑到所有良序集是真类,全体序数$\mathbf{On}$相当于"类的类"。但是能从每个良序型中挑出一个标准的良序集,从而得到了更精确的描述。

\begin{Def}[序数(von Neumann)]
    集合$\alpha. \forall x\in \alpha, x\subset \alpha $ Or $\alpha \subset P(\alpha)$ 称$\alpha $是传递的。

    传递集$\alpha$对于$\in$构成良序集,则称$\alpha $为序数.

    $\emptyset $是序数。$\alpha $是序数$\Rightarrow \alpha \sqcup \{\alpha \}$也是序数
\end{Def}

\begin{Lem}[序数的性质].
    \begin{enumerate}
        \item $\alpha $是序数, $\beta \in \alpha \Rightarrow \beta$ 也是序数
        \item 任意两个序数$\alpha ,\beta $,若$\alpha \subsetneq \beta \Rightarrow \alpha \in \beta $
        \item 任意两个序数$\alpha ,\beta $ 必有 $\alpha \subset \beta \vee \beta  \subset \alpha $ 
    \end{enumerate}
    \begin{proof}
        略
    \end{proof}
\end{Lem}
\begin{Them}[后继、极限序数\label{The order on On}]定义$\mathbf{On}$为序数构成的类
    \begin{itemize}
        \item 对序数定义$\beta < \alpha \Leftrightarrow \beta \in \alpha $. 这是$\mathbf{On}$上的一个全序.$\forall \alpha , \alpha =\{\beta :\beta <\alpha \}$.
        \item 若$C$是由序数组成的类,$C\neq \emptyset \Rightarrow \inf C:=\bigcap C$也是序数,且$\inf C \in C$. $\alpha \sqcup \{\alpha \}=\inf\{\beta : \beta > \alpha \}$.
        \item 若$S$是一个由序数组成的集合,$S\neq \emptyset \Rightarrow \sup S:=\bigcup S$也是序数.
    \end{itemize}
    给定序数$\alpha. \alpha +1:=\alpha \sqcup \{\alpha \}>\alpha $,称为$\alpha $的后继,是大于$\alpha $的最小序数。

    若$\alpha $不是任何序数的后继,则$\alpha =\sup\{\beta :\beta <\alpha \}$, 称为极限序数。

    约定$sup \emptyset=\emptyset$使得零序数$\emptyset $是极限序数
\end{Them}

\begin{Them}[序数类$\mathbf{On}$是真类].
    \begin{proof}
        若$\mathbf{On}$是集合,则$\sup \mathbf{On}$有定义,且序数$\sup \mathbf{On}+1$严格大于所有序数,包括自己。
    \end{proof}

    这称为 Burali-Forti悖论(1897).
\end{Them}

\begin{Exap}[无穷序数$\omega$的构造].
    考虑序数$0:=\emptyset$,并定义一系列后续$1,2,3,...$。

    根据无穷公理中归纳集的概念, $x$是归纳集,$\alpha :=\{y\in x:y \subset x, y\in \mathbf{On}\}$.

    验证:
    \begin{enumerate}
        \item $\emptyset \in \alpha \Rightarrow \alpha \neq \emptyset$
        \item $\alpha$ 是序数
        \item $y\in \alpha \Rightarrow y+1 \in \alpha$
    \end{enumerate}
    因此$\alpha$确实为极限序数。
    
    取$\omega:=\inf\{\text{非零极限序数}\}$.

    满足$n<\omega$的序数称为有限序数。
\end{Exap}

    同构意义下,$\omega = \mathbb{Z}_{\geq 0} $. 
    
    准确的说$\omega$连同其中的后继运算$n\rightarrow n+1$满足Peano的算术公理,后继运算对应于:数学归纳法。

\section{超穷递归及其应用}

定理\ref{The order on On}表明$\mathbf{On}$上的$\leq $也有良序性质:任何序数构成的类必有极小元。这可以得到超穷归纳法。

\begin{Them}[超穷归纳法]$C$是一个序数构成的类.假设
    \begin{itemize}
        \item $0\in C$
        \item $\alpha \in C \Rightarrow \alpha +1 \in C$
        \item 设$\alpha $是极限序数,$\forall \beta <\alpha \rightarrow \beta \in C \Rightarrow \alpha \in C$
    \end{itemize}
    那么$C=\mathbf{On}$.
    如果考虑小于某个$\theta$的序数而非$\mathbf{On}$,断言依然成立。
    \begin{proof}
        假设相反,取不在$C$中的最小序数$\alpha \neq 0$,无论是它的后继或者极限序数都导致矛盾。
    \end{proof}
\end{Them}
\begin{Rmk}[与经典数学归纳法的关系].
    取$\theta = \omega$就得到了经典的数学归纳法,此时不涉及情况iii.

    超穷归纳法通常递归构造一列以序数枚举的数学对象,它基于某规则$G$:
    \begin{itemize}
        \item 第零项 $a_0=G(0)$给定;
        \item 后继项 $\alpha = \beta + 1$, 已知$a_{\beta}$,则可以确定$a_{\alpha}=a _{\beta+1}=G(a_{\beta })$
        \item 极限项 $\alpha $是极限序数,已知$\{a_{\beta}:\beta < \alpha\}$,则可以确定$a_{\alpha}=G(\{a_{\beta}:\beta <\alpha\})$
    \end{itemize}
    这将唯一确定一列集合$a_{\alpha}$.可以考虑$G:\mathbf{V}\rightarrow \mathbf{V}$. 虽然$\mathbf{V}$是真类,但$G$可以解释为一阶逻辑中的某类公式而不影响引入$\theta$-列的概念:这是指函数$a:\theta \rightarrow \mathbf{V}$, 也可以理解成形如$\{a_{\alpha}:\alpha <\theta\}$的列。
\end{Rmk}

\begin{Them}[超穷递归原理].
    $\forall \text{序数}\theta, \exists !\theta \text{-列} \Rightarrow \alpha <\theta \rightarrow a(\alpha)=G(a|_{\alpha})$.
    
    特别地,存在唯一函数:$a:\mathbf{On}\rightarrow \mathbf{V}$ 使得每个序数$\alpha, a(\alpha)=G(a|_{\alpha})$

    几点说明:
    \begin{itemize}
        \item 替换公理模式确保函数$a|_{\alpha }$可以采用等同于集合的限制定义
        \item 函数限制$a|_0$理解为空集0, 按定义$a(0)=G(0)$
        \item 定理前半段只要求$G$对于形如$\alpha \rightarrow \mathbf{V}$的函数有定义,其中$\alpha < \theta$.
    \end{itemize}
    \begin{proof}
        略。
    \end{proof}
\end{Them}
借助超穷递归原理,能对序数定义类似非负整数的运算。
\begin{itemize}
    \item 加法:
    \begin{enumerate}
        \item $\alpha +0:=\alpha $
        \item $\alpha + (\beta +1)=(\alpha +\beta )+1$
        \item $\alpha +\beta =\sup\{\alpha +\xi:\xi < \beta\}$
    \end{enumerate}
    \item 乘法:
    \begin{enumerate}
        \item $\alpha \cdot 0:=0$
        \item $\alpha \cdot (\beta + 1)=(\alpha \cdot \beta) + \alpha $
        \item $\alpha \cdot \beta =\sup \{\alpha \cdot \xi : \xi < \beta \}$
    \end{enumerate}
    \item 指数:
    \begin{enumerate}
        \item $\alpha^0 :=1$
        \item $\alpha^{\beta +1}=\alpha^{\beta }\cdot \alpha $
        \item $\alpha^{\beta}=\sup\{\alpha^{\xi}: \xi < \beta \}$
    \end{enumerate}
\end{itemize}
    可以验证序数对加法和乘法满足结合律,但一般不满足交换律。

\begin{Them}[良序集与序数之间必存在唯一同构].
    $\forall \text{良序集}P,\exists !\text{序数}\alpha$和两序集之间的同构$\phi : P \rightarrow^{\thicksim}\alpha$
    \begin{proof}
        ???
    \end{proof}
\end{Them}

Zermelo良序定理和Zorn引理都依赖于选择公理\ref{Axiom of Choose}.

\begin{Them}[Zermelo 良序定理(1904)\label{Them Zermelo}] 任意集合都能被赋予良序.
    \begin{proof}
        ???
    \end{proof}
\end{Them}

\begin{Them}[Zorn 引理] 非空偏序集中的每个链都有上界,则此集合必有极大元
    \begin{proof}
        ???

        设$P$不含有极大元,$\forall a_0 \in P$,使用超穷递归对每个序数$\alpha$定义$a_{\alpha}\in P \rightarrow \alpha <\beta \Rightarrow a_{\alpha}<\alpha_{\beta}$.

        ...

        可以导出$\mathbf{On}$也是集合。
    \end{proof}
    
\end{Them}

\section{基数}

基数是描述集合大小的标准,可以从等势的概念出发,后面联系到序数
\begin{Def}[集合间的等势].

    若两个集合$X,Y$之间存在双射$\phi :X\rightarrow Y,$称$X,Y$等势
\end{Def}
    等势构成了集合间的等价关系,集合$X$的势记作$|X|$.若存在单摄$\phi :X\rightarrow Y$,记作$|X| \leq |Y|$.
    
    关系$\leq $只依赖于等势类,并满足传递性:$|X|\leq |Y|,|Y|\leq |Z| \Rightarrow |X| \leq |Z|$
\begin{Them}[Schroder-Bernstein\label{Them Schroder-Bernstein}] 两集合$X,Y$. 满足$|X|\leq |Y| \wedge |Y| \leq |X|\Rightarrow |X|=|Y|$.
    \begin{proof}
        ???构造成序列,用递归的方法把两个单射组合成一个双射
    \end{proof}
\end{Them}
此定理表明基数上的关系$\leq $具有反称性,因此在等势类上定义了偏序。

等势类的基本运算:
\begin{itemize}
    \item $|X|+|Y|=|X \sqcup Y|$
    \item $|X|\cdot |Y|= |X \times Y|$
    \item $|X|^{|Y|}=|X^Y|$
\end{itemize}
注意到$2^{|X|}=|P(X)|$.

\begin{Them}[Cantor] 任意集合$X, |P(X)|>|X|$
    \begin{proof}
        存在单射$X\rightarrow P(X)$,然而任意映射$\phi :X\rightarrow P(X)$都不是满射:验证集合$Y:=\{x\in X: x\notin \phi(X)\} \notin \text{im} \space \phi$
    \end{proof}
\end{Them}

\begin{Def}[基数与序数的关系]基数是一个等势类中的最小序数

    序数$\kappa$称为基数。如果对于任意序数$\lambda < \kappa \Rightarrow |\lambda|<|\kappa|$
\end{Def}

以下使用选择公理联系等势类和基数。
\begin{Prop}[任意集合的基数(等势类)都存在序数与之一一对应].

    $\forall \text{集合}X$, 取最小的序数$\alpha(X)\rightarrow |X|=|\alpha(X)|$.则$X\rightarrow \alpha(X)$给出了 等势类和基数的一一对应。

    特别地,$\forall \text{集合}X,Y\Rightarrow |X|\leq |Y| \vee |Y|\leq |X|$.
    \begin{proof}
        序数$\alpha(X)$的存在性起源于Zermelo良序定理\ref{Them Zermelo}, 它是基数,其余断言是显然的???
    \end{proof}
\end{Prop}
    有限集是基数为有限序数的集合,可数集是基数为$\omega$的集合,其余集合称为不可数集。
    注意到任何无穷集合包含$\mathbb{Z}_{\geq 0}$等势的子集,这是因为无穷序数包含了$\omega$。
\begin{Lem}[].
    任意序数都有基数上界;
    基数组成的集合$S, \sup{S}$也是基数
    \begin{proof}
        集合及其子集上所有可能的良序结构形成了一个集合,$\mathbf{On}$是真类,因此$\forall \text{集合} X,\exists \text{序数}\theta \rightarrow |\theta |>|X|$,取$X=\alpha$得到命题1.???

        假设$\exists \text{序数}\beta< \alpha:=sup{S} \rightarrow |\beta |=|\alpha |$.根据上确界性质$\exists \kappa \in S\rightarrow \kappa > \beta$.
        因此$|\beta|\leq |\kappa|\leq |\alpha |=|\beta|\rightarrow |\beta |=|\kappa|$,因此$\kappa$不是基数,矛盾。

        利用基数是最小等势类的序数。
    \end{proof}
\end{Lem}
    任意无穷集$\alpha$都满足$|\alpha \sqcup \{\alpha \}|=|\alpha |$(在$\mathbb{Z}_{\geq 0}$内部处理),所以无穷序数必为极限序数。无穷基数称作 $\aleph$数。上述引理给出了用序数枚举无穷基数的方法

    递归定义:
    \begin{itemize}
        \item $\aleph_0:=\omega$ 最小的无穷基数
        \item $\aleph_{\alpha+1}:=$大于$\aleph_{\alpha}$的最小基数
        \item $\aleph_{\alpha}:=\sup_{\beta <\alpha} \aleph_{\beta}$
    \end{itemize}
    特别地,基数全体构成一个真类。
\begin{Exap}[$|\mathbb{R}|=2^{\aleph_0}$ Cantor].
    \begin{proof}
        首先$|\mathbb{Q}|=|\mathbb{Z}_{\geq 0}|=\aleph_0$. 使用Dedekind分割 $x\rightarrow \{r\in \mathbb{Q}:r<x\}$给出了$\mathbb{R}\rightarrow P(\mathbb{Q})$上的单射。因此$|\mathbb{R}|\leq |P(\mathbb{Q})|=2^{\aleph_0}$

        反之考虑区间$[0,1]$上的Cantor集
        $$C:=\{\sum^{\infty}_{n=1}{a_n 3^{-n}}:\forall n,a_n=0 \vee 2\}$$
        形如上式的三进制表示法是唯一的,因此$|\mathbb{R}|\geq |C|=2^{\aleph_0}$
    \end{proof}
\end{Exap}

集合论中常称实数集为"连续统"。著名的连续统假设断言$2^{\aleph_0}=\aleph_1$.已知若预设ZFC公理系统的一致性,那么连续统假设成立(P. Cohen, 1963)或否定(K. Godel, 1940)都无法从ZFC的公理推出。

\begin{Them}[$\alpha \times \alpha $上的典范良序].
    真类$\mathbf{On}^2=\mathbf{On}\times \mathbf{On}$上存在良序$\prec$使得对每个序数$\alpha$皆有
    $\mathbf{On}^2_{\prec(0,\alpha)} =(\alpha \times \alpha, \prec)$. 称作$\alpha \times \alpha $上的典范良序。进一步,$(\aleph_{\alpha}\times\aleph_{\alpha}, \prec)$上的序型为$\aleph_{\alpha}$; 作为推论:$\aleph_{\alpha}\cdot \aleph_{\alpha}=\aleph_{\alpha}$.

    \begin{proof}
        在$\mathbf{On}$上定义序关系$\prec. (\alpha ,\beta )\prec(\alpha', \beta')$
        \begin{itemize}
            \item $\max\{\alpha, \beta \}<\max\{\alpha',\beta'\}$;
            \item $\max\{\alpha, \beta \}=\max\{\alpha', \beta'\}$,以字典序比较大小。
        \end{itemize}
        此时$\prec$是$\mathbf{On}^2$上的良序。且$0$是最小序数,故$(\alpha', \beta')\prec(0,\alpha)\Leftrightarrow \alpha',\beta'<\alpha$,得到$\mathbf{On}^2_{(0,\alpha)}=\alpha \times \alpha$.

        假设$\alpha$为使得$(\aleph_{\alpha}\times \aleph_{\alpha}, \prec)$不同构于 $\aleph_{\alpha}$的最小序数。因此$\alpha >0$,但是可以根据Cantor对角线手法给出
        $(\aleph_0\times \aleph_0, \prec)\simeq \aleph_0$

        另序数$\gamma$表示$(\aleph_{\alpha}\times\aleph_{\alpha},\prec)$的序型,取同构$f:\gamma\simeq(\aleph_{\alpha}\times\aleph_{\alpha}, \prec)$.基数的性质给出
        \[
            \gamma \geq |\gamma| = |\aleph_{\alpha}\times \aleph_{\alpha}|\geq \aleph_{\alpha}
        \]
        $\alpha$的选取表示$\gamma \neq \aleph_{\alpha} \Rightarrow \gamma > \aleph_{\alpha}$. $\aleph_{\alpha}$是$\gamma$的真前段,限制$f$就得到从$\aleph_{\alpha}$到$(\aleph_{\alpha}\times\aleph_{\alpha},\prec)$的某个真前段的同构。根据$\gamma$的定义,存在序数$\sigma < \aleph_{\alpha}$使得$f(\aleph_{\alpha})\subset \sigma \times \sigma$.$\alpha, \sigma$都是无穷。
        根据基数定义,$\exists \beta <\alpha \Rightarrow |\sigma|=\aleph_{\beta}$.$\alpha$ 的极小性假设蕴涵$\aleph_{\beta}\times \aleph_{\beta}=\aleph_{\beta}$(良序)
        \[
            \aleph_{\alpha}=|f(\aleph_{\alpha})|\leq |\sigma|\times|\sigma|=\aleph_{\beta}\cdot\aleph_{\beta}=\aleph_{\beta}
        \]
        $\rightarrow \beta < \alpha\Rightarrow \aleph_{\beta}<\aleph_{\alpha}$,矛盾。
    \end{proof}
\end{Them}

\begin{Them}[基数运算的性质].
    \begin{itemize}
        \item $\kappa + \lambda = \kappa \cdot \lambda = \max\{\kappa, \lambda\}$
        \item $2\leq \kappa \leq \lambda \Rightarrow \kappa^\lambda=2^\lambda$
    \end{itemize}
    \begin{proof}
        设$\lambda$为无穷且$\lambda\geq\kappa>0$,使用Schroder-Bernstein定理\ref{Them Schroder-Bernstein},简化为证明
        \begin{enumerate}
            \item 设$\lambda$无穷而且$\lambda\geq \kappa>0$
            \[
                \lambda \leq \kappa + \lambda \leq \kappa \cdot \lambda \leq \lambda \cdot \lambda
            \]
            $\lambda=\lambda\cdot\lambda$.显然。

            \item $2^{\lambda}\leq \kappa^{\lambda}\leq (2^{\lambda})^{\lambda}=2^{\lambda\cdot\lambda}=2^{\lambda}$ 显然。
        \end{enumerate}
    \end{proof}
\end{Them}

\begin{Def}[正则基数].
    无穷基数$\alpha$称为正则基数,如果不存在极限序数$\beta <\alpha$和严格增的序数列$\{\alpha_{\xi}:\xi<\beta\}$使得$\sup{\{\alpha_{\xi}:\xi<\beta\}}=\alpha$
\end{Def}
    正则基数是无法由更小的基数拼凑而成的无穷基数。(得用序数拼凑)
\begin{Exap}[非正则基数的例子].

    $\gamma$是任意序数,形如$\aleph_{\gamma+\omega}$的基数均非正则。取$\beta:=\gamma +\omega$和$\alpha_{\xi}:=\aleph_{\xi}$. 关系$\alpha:=\aleph_{\beta}>\beta$.
    $\aleph$数的定义表明$\sup \{a_{\xi}:\xi < \beta\}=\sup \{\aleph_{\gamma + n}:n< \omega\}=\aleph_{\gamma + \omega}$
\end{Exap}

\section{Grothendieck宇宙}

\begin{Def}[宇宙]集合$\mathcal{U}$,满足五性质
    \begin{enumerate}
        \item[U.1] $u\in \mathcal{U}\Rightarrow u \subset \mathcal{U}$: $\mathcal{U}$是传递集
        \item[U.2] $u,v\in \mathcal{U}\Rightarrow \{u,v\}\in \mathcal{U}$
        \item[U.3] $u\in \mathcal{U}\Rightarrow P(u)\in \mathcal{U}$ 
        \item[U.4] $I\in U, \{u_i:i\in I\}, \forall i, u_i\in \mathcal{U}\Rightarrow \bigcup_{i\in I}u_i \in \mathcal{U}$
        \item[U.5] $\mathbf{Z}_{\geq 0}\in U$ 或者$\emptyset \in \mathcal{U}$ 
    \end{enumerate}
\end{Def}
    给定集合$X$,若$X\in \mathcal{U}$称为$\mathcal{U}$-集

    若$X$和一个$\mathcal{U}$-集等势,称为$\mathcal{U}$-小集

    给定宇宙$\mathcal{U}$,以下四结论成立:
    \begin{itemize}
        \item $u \subset v \in \mathcal{U}\Rightarrow u\in \mathcal{U}$
        \item $u\in U\Rightarrow \bigcup u =\bigcup_{x\in u}x\in \mathcal{U}$
        \item $u,v\in \mathcal{U}\Rightarrow u\times v \in \mathcal{U}$
        \item $I\in \mathcal{U}$,集族$\{u_i:i\in I\}$, 满足$\forall i,u_i \in \mathcal{U}\Rightarrow \prod_{i\in I}u_i \in \mathcal{U}$
    \end{itemize}
    宇宙的概念用于在$\mathcal{U}$内可以进行大部分常见的数学操作而不涉及真类,这是集合论问题的防火墙。

\begin{Asum}[A. Grothendieck\label{Asum Grothendieck Set in Universe}]任意集合都有包含该集合的宇宙
    ???
\end{Asum}

\begin{Prop} 任意集合都属于某个由序数枚举的集合$V_{\alpha},\alpha$是序数
    \begin{proof}
        ???
    \end{proof}
\end{Prop}

\begin{Def}[强不可达基数]基数$\kappa$满足三性质
    \begin{enumerate}
        \item[I.1] $\kappa$不可数
        \item[I.2] $\kappa$是正则基数
        \item[I.3] $\forall$基数$\lambda$,$\lambda<\kappa \Rightarrow 2^{\lambda}<\kappa$  
    \end{enumerate}
\end{Def}
    可以证明I.3蕴含$\lambda,\nu <\kappa\Rightarrow \lambda^\nu < \kappa$.
    强不可达基数是现代集合论着力研究的大基数的一员。Bourbaki证明了以下结果:

\begin{Them}[宇宙是层垒谱系中强不可达基数的序数成员].

    宇宙是形如$V_{\kappa}$的成员,其中$\kappa$是一个强不可达基数
\end{Them}
    强不可达基数的性质告诉我们:从$V_{\kappa}$的元素出发,在ZFC系统内无论怎么操作都不会超过$V_{\kappa}$.

    对于强不可达基数$\kappa$,宇宙$\mathcal{U}:=V_{\kappa}$和从属关系$\in$构成了一个$ZFC$的模型,相当于在集合论内部虚拟地运行了一套集合论。

    $V_{\kappa}$的元素若看作“集合”,则使用模型$(V_{\kappa},\in)$考察“类”就是$V_{\kappa+1}=P(V_{\kappa})$的元素。

\begin{Rmk}在ZFC内无法证明强不可达基数的存在性;也不能要求仅有一个强不可达基数。

    假设\ref{Asum Grothendieck Set in Universe}等价于对任意基数$\lambda$存在强不可达基数$\kappa$,这个条件非常强,对于大多数的范畴论构造和数学论证,最多只要求存在某个宇宙$\mathcal{U}$,这表明大多数情况下仅需要单个强不可达基数。但这个条件也是不在ZFC内的。
\end{Rmk}

\section{习题}