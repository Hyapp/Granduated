\section[1]{Introductory Ideas}
The definitions will be refer to all book, but 1.8 is refered to only in Section 3.5.

\textbf{Throughout the book, mapping symbols are written on the right.}

\subsection[1]{Basic Definitions}

\begin{Def}[Semigroup]
    A groupoid $(S,\mu), S \neq \emptyset, \mu$ is a map $: S\times S \to S$ and $\mu$ is associative
    \begin{equation}
        \forall x,y,z \in S, ((x,y)\mu,z)\mu (x,(y,z)\mu)\mu
    \end{equation}
\end{Def}

The notation of operator $\mu$ could be notated as multiplication.

$$(x y)z=x(y z)$$

When the multiplication of semigroup is clear from the context, we shall write simply $S$ rather that $(S,.)$

\begin{Def}[Order of Set]
    The cardinal number of set S, $|S|$.
\end{Def}

\begin{Def}[Commutative (abelian) semigroup]
    $\forall x,y \in S, x y = y x$
\end{Def}

\begin{Def}[Identity]
    $1\in S, \forall x \in S \to 1x=x1=x $
\end{Def}

$S$ has at most one identity element
$$\forall x \in S, x1'=1'x=x \to 1'=11' =1$$

\Def[Monoid]{$S, 1 \in S$}

\begin{Sym}[$S^{1}$]
    \[
        S^1 =
        \begin{cases}
            S               &   \text{if } 1 \in S  \\
            S \cup \{1\}    &   \text{if } 1 \notin S
        \end{cases}
    \]
\end{Sym}

\begin{Def}[Zero Element]
    A semigroup S. $|S| > 1. \forall x \in S, 0x=x0=0$ 
\end{Def}

It's easy to add a 0 to semigroup.

\begin{Exap}[Trivial Semigroups]

\end{Exap}

\begin{Sym}[$S^{0}$]
    \[
        S^0 =
        \begin{cases}
            S               &   \text{if } 0 \in S  \\
            S \cup \{0\}    &   \text{if } 0 \notin S
        \end{cases}
    \]
\end{Sym}

\begin{Rmk}
    The semigroup's extentions of 1 and 0 is not work on group.
    $G$ is a group, $G \cup \{0\}$ is a semigroup but not a group yet. 
\end{Rmk}

\begin{Def}[Left/Right zero semigroup]
    \[
        \text{Left: \space} S \neq \emptyset, \forall a,b \in S, ab=a
    \]
    Right is analogue.
\end{Def}

\begin{Exap}[Semigroups]
    \[
        I=\left[0,1\right],xy=\min(x,y)
    \]   
    \[
        \{0\},0 \text{\space is both Identity and Zero}
    \]
\end{Exap}

\begin{Def}[Multiplication of Set]
    \[
        AB=\{ab:a \in A \wedge b \in B\}
    \]    
    Notes: $A^2 \neq \{a^2 : a \in A \} $

    \[
        Ab=A\{b\}
    \]
\end{Def}

\begin{Exap}[Monoid and Multiplication]
    \[
        1 \notin S \rightarrow 1 \notin S^1
    \]
    \[
        1 \notin 
        \begin{cases}
            S^1 a & = Sa \cup \{a\} \\
            a S^1 & = aS \cup \{a\} \\
            S^1 a S^1 & = SaS \cup Sa \cup aS \cup \{a\}
        \end{cases}
    \]
\end{Exap}

\begin{Def}[Group]
    If a semigroup S has the property that
    \[
        (\forall a \in S) aS=S \wedge Sa=S
    \]
    This definition is equivalent to the common definition of group. 
    \begin{proof}
        \begin{gather*}
            Sa=S \rightarrow \exists s \in S, sa=a \rightarrow s \text{\space is the left identity element of }a \\
            \forall x \in S, ax=sax=s(ax) \rightarrow s \text{\space s is a left identity element of } ax   \\
            {ax}=S  \\
            \rightarrow s \text{\space is the left identity element of } S  \\
            \rightarrow \text{analogue, }s \text{\space is the identity element of }S   \\
            \rightarrow E \in S \text{;}\\
            aS=S \rightarrow \exists b \in S, ba=E \\
            \rightarrow b=a^{-1}    \\
            \rightarrow S\text{\space is a group}
        \end{gather*}
    \end{proof}
\end{Def}

\begin{Def}[0-Group]
    $G$ is a group,$G^0=G\cup\{0\}$ is a semigroup.
\end{Def}

\begin{Prop}[There is no $0-group$ different from $G\cup\{0\}$]
    \[
        \text{A semigroup with zero is a }0-group \Leftrightarrow (\forall a \in S \setminus \{0\})aS=S\wedge Sa=S
    \]
    \begin{proof}
        Sufficiency
        \begin{gather*}
        S=G^0,a \in G=S \setminus \{0\} \rightarrow aG=Ga=G \\
        aS=aG \cup a\{0\}=aG \cup \{0\} \\
        Sa=Ga \cup \{0\} a=Ga \cup \{0\}    \\
        \rightarrow aS = Sa = S
        \end{gather*}
        Necessity
        \begin{gather*}
            (\forall a \in S \setminus \{0\})aS=Sa=S    \\
            \text{Let }G=S \setminus \{0\}
            \text{Suppose }\exists a,b \neq 0 \in G \rightarrow ab=0   \\
            \rightarrow S^2=(Sa)(bS)=S(ab)S=S\{0\}S=\{0\}   \\
            \rightarrow S=aS \subset S^2 =\{0\} \\
            \text{It is a contradiction.}   \\
            \rightarrow \forall a,b \in G \rightarrow ab \in G  \\
            \forall a \in G, aG=aS \setminus a\{0\}=aS \setminus \{0\}=S \setminus \{0\}=G  \\
            \forall a \in G,Ga=Sa \setminus \{0\}a=Sa \setminus \{0\}=S \setminus \{0\}=G   \\
            \rightarrow G  \text{\space is a group.}
        \end{gather*}
    \end{proof}
\end{Prop}

\Def[Subsemigroup]{$T \subset S \wedge T \neq \emptyset \wedge \forall x,y \in T \rightarrow xy \in T$ or $T^2 \subset T$}

\Def[Idempotent]{$e \in S, e^2=e$}

\begin{Exap}[Subgroups]
    $\{0\},\{1\},\{e\}$ \space are all subgroups.   
\end{Exap}

\begin{Rmk}[No trivial subgroup's condition]
    \[
        T \subset S,(\forall a \in T) aT=T \wedge Ta=T
    \]
\end{Rmk}

\begin{Def}[Left/Right Ideal, Ideal]
    \[
        A: A \subset S \wedge A \neq \emptyset \space
        \begin{cases}
            SA \subseteq A \text{\space : Right Ideal}  \\
            AS \subseteq A \text{\space : Left Ideal }  \\
        \end{cases}
    \]
    Ideal : Both left and right ideal.
\end{Def}

\Rmk{Every ideal is a subsemigroup, but the converse is not the case.}

\begin{Def}[Proper]
    Ideal : $I: \{0\} \subset I \subset S$. Symbol '$\subset$'  is strictly. 
\end{Def}

\begin{Def}[Morphism.Homomorphism]
    $S,T$ are semigroups.
    \[
        \text{A map }\phi: S \rightarrow T, \forall x,y \in S,(xy)\phi=(x)\phi(y)\phi
    \]
\end{Def}

\begin{Rmk}[Morphism of monoid has better properties]
    \[
        (S,.,1_S),(T,.,1_T)\text{ are monoids.}
        \phi \text{ is a morphism} \rightarrow
        1_S\phi =1_T
    \]   
    \begin{proof}
        \begin{gather*}
            \forall x \in S, (x) \phi = (1_S x) \phi =(1_S) \phi (x) \phi   \\
            \rightarrow (1_S) \phi \text{\space is the left identity of S}  \\
            \text{The right analogue is same.}  \\
            \rightarrow (1_S) \phi = 1_T
        \end{gather*}
    \end{proof}
\end{Rmk}

\begin{Def}[Monomorphism]
    $S,T$ are semigroups, A morphism $\phi : S \rightarrow T \wedge \phi$ is one-one.
    
    This definition is equivalent to the 'categorical' definition of a monomorphism as a right cancellative morphism. (The function symbol in this book is right.)

    \[
        \forall \text{semigroups } U, \forall \text{morphisms } \alpha, \beta : U \rightarrow S, \alpha \phi = \beta \phi \Rightarrow \alpha = \beta
    \]
\end{Def}

\begin{Def}[Isomorphism]
    \[
        \phi. \exists \phi^{-1}: T \rightarrow S. \rightarrow
        \phi\phi^{-1}=I_S \wedge \phi^{-1}\phi=I_T. S \simeq T
    \]
\end{Def}

\begin{Def}[Endomorphism, Automorphism]
    \begin{gather*}
        \text{A morphism } \phi: S \rightarrow S    \text{: Endomorphism}\\
        \text{An Endomorphism } \phi \text{ is onto and one-one \text{: Automorphism}}
    \end{gather*}
\end{Def}

\begin{Def}[Direct(cartesian) product;Projection morphism]
    \[\text{Semigroups:}S,T. S\times T \text{ is } (s,t)(s',t')=(ss',tt')\]

    General notion of direct product.
    \begin{gather*}
        \text{The product of } \{S_i:i\in I\}   \\
        \text{All maps } p: I \rightarrow \bigcup _{i \in I} S_i. ip\in S_i, P=\{p\}  \\
        \text{Define the multiplication }i(pq)=(ip)(iq) \\
        \rightarrow P \text{\space is a semigroup}
    \end{gather*}
    Projection morphism
    $\pi_i:P \rightarrow S, p\pi_i=ip(p \in P)$
    
    Moreover, if $T$ is a semigroup and if there are morphisms $\tau_i:T\rightarrow S_i(i \in I) \rightarrow \exists !\gamma: T \rightarrow P, \forall i\in I,\gamma \pi_i=\tau_i$. The map $\gamma:= \forall t \in T, (i)(t\gamma)=t\tau_i,(i \in I)$

    $P$ is the product of the semigroups $S_i$ said in categroy.
    \begin{Prof} The isomorphism of $P \to \prod S_i$
        \begin{align*}
            & \phi: P \to \prod S_i. p \phi =I p    \\
            & \forall p \in P, p \phi = I p = s \in \prod S_i \\
            \Rightarrow & \forall p_1 \neq p_2, p_1 \phi = I p_1 \neq I p_2 = p_2 \phi  \\
            \Rightarrow & \phi \text{\space is one-one} \\
            & \forall s \in \prod S_I \rightarrow \exists p \in P, I p = s  \\
            \Rightarrow & \phi \text{\space is onto}    \\
            \Rightarrow & \phi \text{\space is bijective}
        \end{align*}
    \end{Prof}
\end{Def}

\begin{Def}[Permutation(Symmetric) group; Full transformation semigroup]Set $X$.
    \[
    \begin{cases}
        (\mathcal{G},\circ):= \text{All permutations of }X  &  \text{Symmetric group}   \\
        (\mathcal{T},\circ):= \text{All maps of }X          &   \text{Full transformation semigroup}
    \end{cases}\]
    
    $\mathcal{G}_X$, consisting of all bijections from X onto X, is a subgroup of $\mathcal{T}_X$.

    $|\mathcal{G}_X|=n!; |\mathcal{T}_X|=n^n$
\end{Def}

\begin{Def}[Transformation semigroup. Representation of S. Faithful Representation]
    \[\begin{cases}
        S \text{ is a subsemigroup of }\mathcal{T}_X   &   \text{Transformation semigroup} \\
        \text{A morphism }\phi: S \rightarrow \mathcal{T}_X &   \text{Representation of S (by maps).}   \\
        \text{A representation of S }\phi \text{ is one-one} &  \text{Faithful ...}
    \end{cases}\]
    
    The first S and second S is different.
\end{Def}

\begin{Them}
    If $S$ is a semigroup and $X=S^1$ then there is a faithful representation $\phi:S \rightarrow \mathcal{T}_X$.
    \begin{proof}
        \begin{gather*}
            \forall a \in S, \rho_a : S^1\rightarrow S^1:= x\rho_a=xa. (x\in S^1)    \\
            \exists \alpha: S\rightarrow\mathcal{T}_X:=a\alpha=\rho_a(a\in S)   \\
            a\alpha=b\alpha \rightarrow \rho_a=\rho_b\rightarrow  \forall x \in S^1,xa=xb\rightarrow 1a=1b \rightarrow a=b  \\
            \rightarrow \alpha \text{\space is one-one} \\
            \forall x \in S^1,x(\rho_a\rho_b)=(x\rho_a)\rho_b=(xa)b=x(ab)=x\rho_{ab} \rightarrow (a\alpha)(b\alpha)=(ab)\alpha  \\
            \rightarrow \alpha \text{\space is a morphism}  \\
            \rightarrow \alpha \text{\space is a faithful representation of }S
        \end{gather*}
    \end{proof}

    The representation $\alpha$ introduced in this proof is called extended right regular representation.
\end{Them}

\Def[Rectangular band]{$\forall a,b \in S, aba=a$}

\begin{Them}
    Let $S$ be a semigroup. Then the four conditions are equivalent.
    \begin{gather*}
        S \text{\space is a rectangular band}   \\
        \forall s \in s, s^2=s; \forall a,b,c \in S \rightarrow abc=ac  \\
        \exists \text{\space left zero ... } L,\text{right zero ... }R\rightarrow S \simeq L \times R   \\
        S\simeq A\times B, A\neq \emptyset,B\neq \emptyset. \text{multiplication: } (a_1,b_1)(a_2,b_2)=(a_1,b_2)
    \end{gather*}

    \begin{Prof}
        \center{$1 \rightarrow 2$}
        \begin{align*}
            &\forall x \in S, x=xxx=x^3\rightarrow xx=x^3x=x^4   \\
            &x^4=x(xx)x=x \rightarrow x=x^4=x^2 \\
            &\rightarrow x \text{\space is idempotent.} \\
            &\forall a,b,c \in S,ac=(aba)(cbc)=a(bacb)c=abc
        \end{align*}
        \center{$2 \rightarrow 3$}
        \begin{align*}
            &\text{Fix an element } c\in S. L=Sc, R=cS\\
            &\forall x=zc,y=tc, x,y\in L \rightarrow xy=zctc=zc^2=zc=x  \\
            &\rightarrow L \text{\space is a left zero ...} \\
            &\rightarrow R \text{\space is a right zero ...}    \\
            &\phi:S\rightarrow L\times R:=x\phi=(xc, cx)(x\in S)    \\
            &(xc,cx)=(yc,cy) \rightarrow x=x^2=xcx=ycx=ycy=y^2=y    \\
            &\rightarrow \phi \text{\space is one-one.}    \\
            &\forall (ac,cb)\in L\times R, (ac,cb)=(abc,cab)=(ab)\phi   \\
            &\rightarrow \phi \text{\space is onto.}    \\
            &\forall x,y \in S  \\
            &(xy)\phi=(xyc,cxy)=(xc,yc)=(xcyc,cxcy)=(xc,cx)(yc,cy)=(x\phi)(y\phi)   \\
            &\rightarrow \phi \text{\space is a morphism.}  \\
            &\rightarrow S\simeq L\times R
        \end{align*}

        \center{$3 \rightarrow 4$}
        \begin{align*}
            &S=L\times R.\text{L is left zero ... and R is right zero ...}  \\
            &\text{Multiplication: } (a,b)(c,d)=(ac,bd)=(a,d)
        \end{align*}
        \center{$4 \rightarrow 1$}
        \begin{align*}
            &S=A\times B, \text{with multiplication: }(a,b)(c,d)=(a,d)\\
            &\forall a=(x,y),b=(p,q)\in S   \\
            &\rightarrow aba=(x,y)(p,q)(x,y)=(x,q)(x,y)=(x,y)=a \\
            &\rightarrow S\text{\space is a rectangular band.}
        \end{align*}
    \end{Prof}
    \[\begin{tikzcd}
        (c,b)
            \arrow[r, dash]
            \arrow[d, dash] &
        (c,d)
            \arrow[d, dash] \\
        (a,b)
            \arrow[r, dash] &
        (a,d)
    \end{tikzcd}\]

\end{Them}

\subsection[2]{Monogenic Semigroups}

\begin{Sym}[$\langle A\rangle $, Generators]
    \[
        A \subset S, U_i \text{\space are all subgroups that }A \subset U_i.\langle A \rangle:=\bigcap_{i \in I}U_i
    \]
    
    $\langle A \rangle$ has two properties:
    \[\begin{cases}
        A \subseteq \langle A\rangle    \\
        \text{Subsemigroup } U, A\subset U \rightarrow \langle A\rangle \subseteq U
    \end{cases}\]

    If $\langle A\rangle = S$ we say that A is a set of generators, or a generating set, of S.
\end{Sym}

\begin{Exap}
    Finite $A$.

    $A=\{a\}. \langle A\rangle=\{a,a^2,a^3,...\}$.

    If we need a submonoid of $S$ generated by $S$, the $A$ always contains $1$.

    $\langle A \rangle=\{1,a,a^2,...\}$
\end{Exap}

\begin{Def}[Monogenic semigroup]
    $S=\langle a\rangle$ is said to be a monogenic semigroup.

    $\langle a\rangle$ is said to be a monogenic subsemigroup of $S$ generated by the element $a$.

    Order of element $a = |\langle a\rangle|$

    The analogue of monogenic in group-theoretic terminology named 'cyclic'. We must judge whether monogenic semigroups are 'round' enough to merit the description 'cyclic'. 
\end{Def}

\begin{Def}[Finite/Infinite order]Period, Index.

    $a\in S, \langle a\rangle=\{a,a^2,...\}$.

    If $a^m=a^n\rightarrow m=n$, that $\langle a\rangle \simeq (N,+)$. We say that $(\langle a\rangle, \cdot)$ is an infinite monogenic semigroup, and a has infinite order in $S$.
    \begin{align*}
        \text{Index: } & \min(\{x \in N: a^x=a^y,x\neq y\})   \\
        \text{Period: }& \min(\{x\in N: a^{m+x}=a^m\})  \\
    \end{align*}
\end{Def}

\begin{Exap}
    Some properties of finite period geenrator.'
    
    $a$ is an element with index $m$ and period $r \rightarrow a^m=a^{m+r}$.
    Moreover, $(\forall q \in N)a^m=a^{m+qr}$.
    
    $\langle a\rangle=\{a,a^2,...,a^m,a^{m+1},...,a^{m+r-1}\}.|\langle a\rangle|=m+r-1$.
\end{Exap}

\begin{Def}[Kernel of $\langle a\rangle$]
    $K_\alpha=\{a^m,...,a^{m+r-1}\}$
\end{Def}

\begin{Prop}
    $K_a$ is a subsemigroup, indeed a ideal, of $\langle a\rangle$. $a^{m+u}$.
    \begin{proof}
        $K_a\langle a\rangle=\langle a\rangle K_a=\{a^{m+1},...,a^{2m+r-1}\}=K_a$
    \end{proof}
    $K_a$ is a subgroup, indeed a cyclic group.
    \begin{proof}
        \begin{align*}
            &ea^x=a^x \rightarrow e=a^{qr}  \\
            &\leftarrow \exists q \rightarrow a^qr \in K_a \\
            &\rightarrow e \in K_a;  \\
            &\forall u,v \in N, \exists x \in N \rightarrow a^{m+u}a^{m+x}=a^{m_+v} \\
            &\leftarrow x\equiv v-u-m\mod{r} \text{\space and } 0\leq x\leq r-1. \\
            &\rightarrow \forall a^x \in K_a, \exists a^y \in K_a \rightarrow a^{x+y}=e \\
            &\rightarrow K_a \text{ is a group}.    \\
            &\\
            &\leftarrow \exists g,0\leq g\leq r-1 \wedge m+g\equiv 1\mod{r} \\
            &\rightarrow i=a^{m+g},K_a={i,i^2,...,i^{r-1}}  \\
            &\rightarrow K_a \text{ is a cyclic group}.
        \end{align*}
    \end{proof}
\end{Prop}

\begin{Exap}Some monogenic semigroups.
    
    $X=\{1,2,...,7\},\alpha = (\begin{matrix}
        1,2,3,4,5,6,7   \\
        2,3,4,5,6,7,5
    \end{matrix}) \in \mathcal{T}_X$
    \begin{align*}
        \alpha^2=(\begin{matrix}
            1,2,3,4,5,6,7   \\
            3,4,5,6,7,5,6
        \end{matrix})   \\
        \alpha^3=(\begin{matrix}
            1,2,3,4,5,6,7   \\
            4,5,6,7,5,6,7
        \end{matrix})   \\
        \alpha^4=(\begin{matrix}
            1,2,3,4,5,6,7   \\
            5,6,7,5,6,7,5
        \end{matrix})   \\
        \alpha^5=(\begin{matrix}
            1,2,3,4,5,6,7   \\
            6,7,5,6,7,5,6
        \end{matrix})   \\
        \alpha^6=(\begin{matrix}
            1,2,3,4,5,6,7   \\
            7,5,6,7,5,6,7
        \end{matrix})   \\
        \alpha^7=(\begin{matrix}
            1,2,3,4,5,6,7   \\
            5,6,7,5,6,7,5
        \end{matrix})   \\
    \end{align*}
    $\alpha$ has index 4 and period 3. $K_\alpha = \{\alpha^4, \alpha^5, \alpha^6\}$.
    \begin{center}
        \begin{tabular}{c|ccc}
                       & $\alpha^4$ & $\alpha^5$ & $\alpha^6$ \\   \hline
            $\alpha^4$ & $\alpha^5$ & $\alpha^6$ & $\alpha^4$ \\  
            $\alpha^5$ & $\alpha^6$ & $\alpha^4$ & $\alpha^5$ \\
            $\alpha^6$ & $\alpha^4$ & $\alpha^5$ & $\alpha^6$ \\
        \end{tabular}
    \end{center}
    $6\equiv 0 \mod{3} \rightarrow \alpha^6$ is the identity, $4\equiv 1\mod{3}\rightarrow \alpha^4$ is the generator. $(\alpha^4)^2=\alpha^5, (\alpha^4)^3=\alpha^6$.
    Visualize $\langle \alpha \rangle$ :
    \[\begin{tikzcd}
        &&&& 6   
            \arrow[ld]  \\
        1 \arrow[r] &
        2 \arrow[r] & 
        3 \arrow[r] &
        4 \arrow[dr] &  \\
        &&&&  5
            \arrow[uu]
    \end{tikzcd}\]
\end{Exap}

\begin{Them}[$\langle a \rangle \simeq (N,+) \text{ or } \{a,a^2,..,a^m,...,a^{m+r-1}\}$] is all.

    1. $\{x:a^x=a^y,x\neq y\} = \emptyset \rightarrow \langle a\rangle \simeq (N,+)$;

    2. $\exists$ index $m$, period $r$ with:
    \begin{align*}
        &a^m=a^m+r   \\
        &\forall u,v \in N^0, a^{m+u}=a^{m+v} \Leftrightarrow u \equiv v \mod{r}    \\
        &\langle a\rangle =\{a,a^2,...,a^{m+r-1}\}  \\
        &K_a=\{a^m,...,a^{m+r-1}\} \text{ is a cyclic subgroup of }\langle a\rangle
    \end{align*}
\end{Them}

\begin{Rmk}[Finite semigroup]$\langle a\rangle \simeq \langle b\rangle$

    It is easy to see $\langle a\rangle \simeq \langle b\rangle \Leftrightarrow $ they have same index and period.
    We note monogenic semigroup $M(m,r)$ with index $m$ and period $r$.
    $M(1,r)$ is the cyclic group of order $r$.

    Periodic semigroup: $\forall a \in S, |\langle a \rangle|$ is finite.

    \emph{Finite} semigroup is always periodic.
\end{Rmk}

\begin{Prop}Every periodic semigroup has a idempotent.
    
    In a periodic semigroup every element has a power which is idempotent. 
    Every periodic semigroup--in particular, in evert finite semigroup, there is at least one idempotent.
    \begin{proof}
        $\forall a \in S \rightarrow |\langle a\rangle| < \infty\rightarrow \exists I \in K_a$
    \end{proof}
    Otherwise, the idempotent may not exist.
\end{Prop}

\subsection[3]{Ordered Sets, Semilattices and Lattices}

\begin{Def}[Order, Partial Order]A binary relation $\omega$ on set X. If 
    \begin{align*}
        1. & \text{reflexive} & \forall x \in X,(x,x) \in \omega   \\
        2. & \text{antisymmetric} & \forall x,y\in X,(x,y)\in \omega \wedge (y,x)\in \omega \rightarrow x=y    \\
        3. & \text{transitive} & \forall x,y,z \in X, (x,y)\in \omega \wedge (y,z)\in \omega \rightarrow (x,z)\in \omega
    \end{align*}
    Traditionally one writes $x\leq y$ rather than $(x,y)\in \omega$.
\end{Def}
\begin{Def}[Total order] Partial order $(X,\omega)$
    $(\forall x,y\in X)x\leq X \vee y\leq x$   
\end{Def}

\begin{Def}[Minimal, Minimum]
    \begin{align*}
        a\text{ :minimal } & (\forall y\in Y)y\leq a \Rightarrow y=a   \\
        b\text{ :minimum } & (\forall y \in Y)b\leq y
    \end{align*}
    In patrial ordered set it is perfectly possible to have minimal elements that are not minimum.
\end{Def}

\begin{Prop}[Let $Y\neq \emptyset, Y \subset X, (X,\leq)$]Then
    \begin{align*}
        Y \text{ has at most one minimum element.}  \\
        Y \text{ is totally ordered} \rightarrow \text{ 'minimal'='minimum'.}
    \end{align*}
\end{Prop}

\begin{Def}[Minimal condition, Well-ordered] $(X,\leq)$

    Minimal condition: Every non-empty subset of X has a minimal element.

    Well-ordered: A totally ordered set X with minimal condition.
\end{Def}
\Def[Analogue, Maximal, Maximum, Maximal condition]{.}
\begin{Def}[Lower bound, the Greatest lower bound(meet)]$Y\subset X, Y\neq \emptyset$
    \begin{align*}
        \text{Lower bounds: } & \{c\in S, \forall y\in Y, c\leq Y\}\\
        \text{meet: }   & \text{ maximum element of {\{c\}}}
    \end{align*}
    Note: meet $d=\bigwedge \{y:y\in Y\}$. If $Y=\{a,b\}, d=a\wedge b$.
\end{Def}

\begin{Def}[Upper bound, the least upper bound(join)]
    analogue
\end{Def}

\begin{Def}[Lower semilattice, Complete lower semilattice]
    \begin{align*}
        \text{lower semilattice: } & \forall a,b \in X, a \wedge b \in X    \\
        \text{complete lower ...: } & \text{lower ...}, \forall Y \subset X, \exists\bigwedge \{y:y\in Y\}
    \end{align*}
\end{Def}

\begin{Def}[Upper semilattice, Complete upper semilattice]
    analogue    
\end{Def}

\begin{Def}[lattice, complete lattice, sublattice]
    \begin{align*}
        \text{lattice : } & X \text{ both upper and lower semilattice.} \\
        \text{complete ... : } & X \text{both complete upper and complete lower semilattice.}   \\
        \text{sublattice : } & Y\subset X,Y\neq \emptyset. \forall a,b\in Y, a\wedge b, a\vee b \in Y
    \end{align*}
    Note: lattice: $X=(X,\leq, \wedge,\vee)$
\end{Def}

\begin{Prop}[The multiplication and $\wedge$ of lattice].

    Let $(E,\leq)$ be a lower semilattice. Then $(E,\wedge)$ is a commutative semigroup consisting entirely of idempotents, and 
    $$(\forall a,b\in E)a\leq b \Leftrightarrow a\wedge b=a$$

    Conversely, suppose that $(E,.)$ is a commutative semigroup of idempotents. Then the relation $\leq$ on $E$ defined by
    $$a\leq b \Leftrightarrow ab = a$$
    is a partial order on $E$, with respect to which $(E,\leq)$ is a lower semilattice.
    
    \begin{proof}
        1.
        \begin{align*}
            &\forall a,b \in E, a\wedge b \in E    \\
            &\forall a,b,c\in E, a\wedge(b\wedge c)=(a\wedge b)\wedge c  \\
            &\rightarrow E \text{ is a semigroup.}  \\
            &\forall a,b \in E, a\wedge b=b\wedge a \\
            &\rightarrow E \text{ is commutative.}  \\
            &\forall a \in E, a\wedge a=a   \\
            &\rightarrow a \text{ is a idempotent.}  \\
        \end{align*}
        2.
        \begin{align*}
            &\text{By the definition. } a^2=a \rightarrow a \leq a  \\
            & a\leq b \wedge b\leq a \rightarrow ab = a\wedge ba =b\rightarrow a=ab=ba=b    \\
            & a\leq b \wedge b\leq c \rightarrow ab=a\wedge bc =b \rightarrow ac =(ab)c=a(bc)=ab=a  \\
            &\rightarrow a \leq c   \\
            &\rightarrow \leq \text{ is a partial order.}\\
            & \\
            &\forall a,b \in E, ab \leq a \wedge ab \leq b \rightarrow ab\text{ is a lower bound.}  \\
            &\forall c\leq a,c\leq b, c(ab)=(ca)b=cb=c \rightarrow c \leq ab    \\
            &\rightarrow ab=a\wedge b   \\
            &\rightarrow E \text{ is a lower lattice.}
        \end{align*}
    \end{proof}
    This proposition is that the notions of 'lower semilattice' and 'commutative semigroup of idempotents' are equivalent.
\end{Prop}

\begin{Sym}[Hasse diagrams]
    The bigger element is always upper to lower one.If $a\leq b, \forall x\in E, a<x<b$ is impossible, paint a line between $a$ and $b$
    \begin{center}
        \begin{tikzcd}
            & b
            \arrow[ld, dash] \\
            a
        \end{tikzcd}
    \end{center}

    This example shows a lower semilattice. Next table should be used by pattern matching, like computer programming language.
    \begin{center}
        \begin{tabular}{c|c}
        & $x\in \{a,b,c,d,e,f,g,h\}, x \bigwedge y = y \bigwedge x$    \\
        \begin{tikzcd}
            a
            \arrow[rd, dash] & & 
            b
            \arrow[ld, dash]    \\
            & c
            \arrow[d, dash] 
            & \\
            & d
            \arrow[ld, dash]
            \arrow[d, dash]
            \arrow[rd, dash] &   \\
            e
            \arrow[rd, dash] &
            f
            \arrow[d, dash] &
            g
            \arrow[ld, dash] \\
            & h &
        \end{tikzcd}
        &
        \begin{tabular}{c|c} \hline
            $x \bigwedge x = x$ & \\ \hline
            $a \bigwedge b = c$ & \\ \hline
            $a \bigwedge x = x$ & $x \neq b$    \\ \hline
            $b \bigwedge x = x$ & $x \neq a$    \\ \hline
            $c \bigwedge x = x$ & $x \notin \{a,b\}$ \\ \hline
            $d \bigwedge x = x$ & $x \notin \{a,b,c\}$ \\ \hline
            $e \bigwedge x = h$ & $x \in \{f, g\}$ \\ \hline
            $f \bigwedge x = h$ & $x \in \{e, g\}$ \\ \hline
            $g \bigwedge x = h$ & $x \in \{e, f\}$ \\ \hline
            $h \bigwedge x = h$ & \\ \hline
            & \\
            & \\ \hline
            \textcolor{purple}{$a \bigvee b = \emptyset$} & \textcolor{purple}{Not a upper semilattice} \\ \hline
        \end{tabular}
    \end{tabular}
    \end{center}
\end{Sym}

\subsection[4]{Binary Relations; Equivalences}

\Def[Equality(diagnoal)  relation]{ $1_X=\{(x,x):x\in X\}$}

\begin{Def}[$\circ$ on $\mathcal{B}_X$]
    \[
        \forall \rho, \sigma \in \mathcal{B}_X, \rho \circ \sigma =\{(x,y)\in X\times X: (\exists z\in X)(x,z)\in \rho \wedge (z,y)\in \sigma\}
    \]
\end{Def}

\begin{Prop}[$(\mathcal{B}_X,\circ)$ is a semigroup]
    $\forall \rho, \sigma, \tau \in \mathcal{B}_X.$
    \begin{align*}
        (x,y) &\in (\rho \circ \sigma) \circ \tau   \\
        &\Leftrightarrow (\exists \in X)(x,z)\in \rho \wedge (z,y)\in \tau  \\
        &\Leftrightarrow (\exists z\in X)(\exists u \in X)(x,u)\in \rho,(u,z)\in \sigma \wedge (z,y) \in \tau   \\
        &\Leftrightarrow (\exists u \in X)(x,u)\in \rho \wedge (u,y)\in \sigma \circ \tau   \\
        &\Leftrightarrow (x,y)\in \rho \circ (\sigma \circ \tau)    \\
        &\rightarrow (\rho \circ \sigma) \circ \tau = \rho \circ (\sigma \circ \tau)
    \end{align*}
\end{Prop}

\begin{Rmk}[The $\circ$ operator keeps order of relations]
    $\forall \rho, \sigma, \tau \in \mathcal{B}_X$
    \[\rho \subseteq \sigma \rightarrow \rho \circ \tau \subseteq \sigma \circ \tau \wedge \tau \circ \rho \subseteq \tau \circ \sigma\]
\end{Rmk}

\begin{Def}[Domain, Image, Converse] in $\mathcal{B}_X$
    \begin{align*}
        \text{Domain: } & \mathrm{dom } \rho=\{x\in X: (\exists y\in X)(x,y)\in \rho\} \\
        \text{Image: } & \mathrm{im } \rho = \{y\in X:(\exists x\in X)(x,y)\in \rho\}  \\
        \text{Converse: } & \rho^{-1}=\{(x,y)\in X\times X: (y,x)\in \rho\}
    \end{align*}
    By this definition, we immediate that, $\forall \rho,\sigma\in \mathcal{B}_X$:
    \begin{align*}
        \rho \subseteq \sigma \rightarrow \mathrm{dom } \rho \subseteq \mathrm{dom }\sigma \wedge \mathrm{im }\rho \subseteq \mathrm{im } \sigma    \\
        \mathrm{dom }\rho^{-1}=\mathrm{im }\rho, \mathrm{im }\rho^{-1}=\mathrm{dom }\rho
    \end{align*}
\end{Def}

\begin{Def}[$x\in X,x\rho, A\subseteq X, A\rho$].
    \begin{align*}
        x\rho=\{y\in X: (x,y)\in\rho\}  \\
        A\rho=\bigcup\{a\rho:a\in A\}
    \end{align*}
\end{Def}

\begin{Def}[Partial map, Restriction, Extension]
    \begin{align*}
        \text{Partial map: } &\phi \in \mathcal{B}_X, \forall x\in \mathrm{dom }\phi, |x\phi|=1 \\
        & \phi, \varphi \text{ are partial maps}, \phi \subseteq \psi    \\
        \text{Restriction: } & \phi \text{ is a restriction of } \psi, \phi = \psi|_{\mathrm{dom }\phi}  \\
        \text{Extension: } & \psi \text{ is a extension of } \phi   \\
    \end{align*}
    Empty relation is also a partial map.
\end{Def}

\begin{Prop}[$\mathcal{P}_X \subset \mathcal{B}_X$] is a subsemigroup.
    \begin{proof}
        Let $\phi ,\psi \in \mathcal{P}_X, (x,y_1),(x,y_2)\in \phi \circ \psi $.
        \begin{align*}
            \exists z_1,z_2\in X &\rightarrow (x,z_1) \in \phi ,(z_1,y_1)\in \psi, (x,z_2)\in \phi ,(z_2,y_2)\in \psi    \\
            &\rightarrow z_1=z_2\rightarrow y_1=y_2  \\
            &\rightarrow \phi \circ \psi \in \mathcal{P}_X
        \end{align*}
    \end{proof}
\end{Prop}

\begin{Prop}\label{Prop:1.4.3}
    $\phi, \psi \in \mathcal{P}_X$
    \begin{align*}
        \mathrm{dom}(\phi \circ \psi) &= [\mathrm{im} \phi \cap \mathrm{dom}\psi]\phi^{-1}     \\
        \mathrm{im}(\phi \circ \psi) &= [\mathrm{im} \phi \cap \mathrm{dom}\psi]\psi   \\
        (\forall x \in \mathrm{dom}(\phi \circ \psi))x(\phi \circ \psi) &= (x\phi)\psi
    \end{align*}
    \begin{proof}
        1.
        \begin{align*}
            &\forall x\in \mathrm{dom}(\phi\circ \psi ) \rightarrow \exists y,z \in X,(x,z)\in \phi ,(z,y)\in \psi \\
            &\rightarrow z\in \mathrm{im}\phi \cap \mathrm{dom} \psi \\
            &(z,x) \in \phi^{-1}    \\
            &\rightarrow x \in z \phi^{-1} \subseteq [\mathrm{im}\phi \cap \mathrm{dom} \psi ]\phi^{-1} \\
            &\text{Conversely}  \\
            &\forall x\in [\mathrm{im}\phi \cap \mathrm{dom}\psi ]\phi^{-1} \\
            &\rightarrow \exists z\in \mathrm{im}\phi \cap \mathrm{dom}\psi \rightarrow x \in z\phi^{-1} \rightarrow (x,z)\in \phi     \\
            &z\in \mathrm{dom}\psi \\
            &\rightarrow \exists y \in X,(z,y)\in \psi \\
            &\rightarrow (x,y)\in \phi \circ \psi   \\
            &\Rightarrow \mathrm{dom}(\phi \circ \psi )=[\mathrm{im}\phi \cap \mathrm{dom}\psi ]\phi^{-1}
        \end{align*}
        2.
        \begin{align*}
            &\forall x \in \mathrm{im}(\phi \circ \psi)\rightarrow \exists (y,x)\in \psi ,\exists (z,y) \in \phi \\
            &\rightarrow y \in \mathrm{dom}\psi \cap \mathrm{im}\phi    \\
            &(y,x)\in \psi \\
            &\rightarrow x \in y\psi \subseteq [\mathrm{im}\phi \cap \mathrm{dom}\psi ]\psi     \\
            &\text{Conversely}  \\
            &\forall x\in \mathrm{im}(\phi \circ \psi)  \\
            &\rightarrow \exists y \in \mathrm{im}\phi \cap \mathrm{dom}\psi \rightarrow x\in y\psi \rightarrow (y,x)\in \psi   \\
            &y \in \mathrm{im}\phi \\
            &\rightarrow \exists z \in X, (z,y) \in \phi    \\
            &\rightarrow (z,x)\in \phi \circ \psi \\
            &\Rightarrow \mathrm{im}(\phi \circ \psi)=[\mathrm{im}\phi \cap \mathrm{dom}\psi]\psi
        \end{align*}
        The above proofs have used no special properties of partial maps.

        3.
        \begin{align*}
            &(x,y)\in \phi \circ \psi \Leftrightarrow \exists z (x,z)\in \phi \wedge (z,y)\in \psi  \\
            &\rightarrow z=x\phi ,y=z\psi ,y=x(\phi \circ \psi) \\
            &\rightarrow x(\phi \circ \psi)=y=z\psi =(x\phi )\psi 
        \end{align*}
    \end{proof}
\end{Prop}

\begin{Def}[Map, Function]
    Partial map $\phi \wedge \mathrm{dom}\phi =X$. 
\end{Def}

\begin{Prop}
    $(\mathcal{T}_X=\{\phi:X\rightarrow X\}, \circ )$ is a subsemigroup of $(\mathcal{B}_X,\circ )$ 
\end{Prop}

\begin{Prop}
    $X\neq \emptyset$
    \begin{align*}
        \text{1. } & \phi \in \mathcal{P}_X \rightarrow \phi^{-1} \in \mathcal{P}_X \Leftrightarrow \phi \text{ is one-one.}    \\
        \text{2. } & \phi \in \mathcal{T}_X \rightarrow \phi^{-1} \in \mathcal{T}_X \Leftrightarrow \phi \text{ is bijective.}  
    \end{align*}

    \begin{proof}.
        \begin{center}
            \begin{tabular}{c c}
                1. & $\Rightarrow$  \\
                & $\phi\in \mathcal{P}_X \rightarrow \phi^{-1} \in \mathcal{P}_X$   \\
                $\Rightarrow$ & $\forall x\phi = y\phi$     \\
                $\Rightarrow$ & $|x\phi| = 1$              \\ 
                $\Rightarrow$ & $|(x\phi)\phi^{-1}| = 1$    \\
                equivalence$\Rightarrow$ & $1_X \subseteq \phi \circ \phi^{-1}$      \\
                $\Rightarrow$ & $\phi \circ \phi^{-1} = 1_X$    \\
                $\Rightarrow$ & $\phi$ is one-one   \\
                & $\Leftarrow$  \\
                & $\phi$ is one-one $\wedge \phi \in \mathcal{P}_X$ \\
                $\Rightarrow$ & $\forall x\phi^{-1} = y\phi^{-1}$   \\
                $\Rightarrow$ & $x\phi^{-1}\phi = y\phi^{-1}\phi$   \\
                $\Rightarrow$ & $x = y (\phi^{-1} \circ \phi = 1_X $                             \\
                $\Rightarrow$ & $|x\phi^{-1}| = 1$  \\
                $\Rightarrow$ & $\phi^{-1} \in \mathcal{P}_X$   \\
                2. & By:1. $\phi$ is one-one. Target is $\Image \phi = X$   \\
                &  $\Rightarrow$    \\
                & $\phi^{-1} \in \mathcal{T}_X$ \\
                $\Rightarrow$ & $\Dom\phi^{-1} = \Image\phi = X$    \\
                $\Rightarrow$ & $\phi$ is bijective.    \\
                & $\Leftarrow$  \\
                & $\phi$ is bijective   \\
                $\Rightarrow$ & $\phi$ is one-one and onto. \\
                $\Rightarrow$ & $\phi^{-1}$ is one-one and onto.    \\
                $\Rightarrow$ & $\phi^{-1} \in \mathcal{T}_X$
            \end{tabular}
        \end{center}
    \end{proof}
\end{Prop}

\begin{Rmk}[Order relation]$\rho $
    \begin{align*}
        \text{reflexive} &\text{    }  1_X \subseteq \rho \\
        \text{anti-symmetric} &\text{    } \rho\cap \rho^{-1} =1_X \\
        \text{transitive}  &\text{    } \rho \circ \rho \subseteq \rho 
    \end{align*}
\end{Rmk}

\begin{Rmk}[Equivalence relation]$\rho $
    \begin{align*}
        \text{reflexive } & \text{    } 1_X \subseteq \rho \\
        \text{symmetric } & \text{    } \rho^{-1}=\rho \\
        \text{transitive } & \text{    } \rho \circ \rho = \rho 
    \end{align*}

    Symmetric$(\rho \subseteq \rho^{-1})$: $\rho \subseteq \rho^{-1}\rightarrow \rho^{-1}\subseteq \rho \rightarrow \rho^{-1}=\rho $. 
    
    Transitive$(\rho^2 \subseteq \rho)$: $\rho =1_X\circ \rho \subseteq \rho \circ \rho \rightarrow \rho \circ \rho =\rho$.
\end{Rmk}

\begin{Rmk}
    \begin{align*}
        &\rho \text{ is an equivalence on }X   \\
        &\Rightarrow  X = \mathrm{dom}1_X \subseteq \mathrm{dom} \rho ; X=\mathrm{im}1_X\subseteq \mathrm{im}\rho   \\
        &\Rightarrow \mathrm{dom}\rho =\mathrm{im}\rho =X
    \end{align*}
\end{Rmk}

\begin{Def}[Partition of $X$]
    A family $\pi=\{A_i \subseteq X:i\in I\}$, if
    \begin{align*}
        1. &\forall i\in I, A_i \neq \emptyset \\
        2. &\forall i,j \in I, \text{either } A_i = A_j \text{ or } A_i \cap A_j = \emptyset \\
        \Leftrightarrow &(A_i=A_j) \vee (A_i \cap A_j = \emptyset)\wedge \lnot [(A_i = A_j) \wedge (A_i \cap A_j) = \emptyset]  \\
        3. &\bigcup\{A_i: i\in I\}=X
    \end{align*}
\end{Def}

\begin{Prop}\label{Prop:1.4.6}
    Partition and equivalence are closely related.

    Let $\rho $ be an equivalence on a set $X$. Then the family 
    \[\Phi(\rho )={x\rho : x\in X} \]
    of subset of $X$ is a partition of $X$.

    Let $\pi = {A_i: i\in I}$ is a partition of $X$, then the relation:
    \[
        \Psi(\pi) = \{(x,y)\in X\times X:(\exists i\in I)x,y\in A_i\}
    \]
    is an equivalence on $X$.
    \[\forall\text{ equivalence } \rho  \text{ on } X, \Psi(\Phi(\rho ))=\rho, \forall \text{ partition } \pi \text{ of } X, \Phi(\Psi(\pi))=\pi\]
    \begin{proof} 
        1. $\rho$ is an equivalence $\to \Phi(\rho)$ is a partition. 
        \begin{center}
            \begin{tabular}{c c}
                1. & $\forall i, A_i \neq \emptyset$    \\
                & $\forall A_i \in \Phi(\rho)$    \\
                Let & $A_i = \emptyset$    \\
                $\Rightarrow$ & $\forall x\in X, x\rho \neq A_i$  \\
                $\Rightarrow$ & $A_i \notin \Phi(\rho)$ Conflict!  \\
                $\Rightarrow$ & $A_i \neq \emptyset $ \\
                2. & Either $A_i = A_j $ or $ A_i \cap A_j = \emptyset$   \\
                & $\exists x \in A_i,$ $\exists y \in A_j$  \\
                $x\rho y\Rightarrow$ & $\exists p,q \in X, (p, x), (q,y) \in \rho$  \\
                $\Rightarrow$ & $(p, q) \in \rho$    \\
                $\Rightarrow$ & $p\rho = A_i = A_j = q\rho$ \\
                $(x,y)\notin \rho\Rightarrow$ & $\exists p,q\in X, (p, x),(q, y)\in \rho$    \\
                $\Rightarrow$ & $p\rho x\not\rho y \rho q$ \\
                $\Rightarrow$ & $(p, q)\notin \rho$  \\
                $\Rightarrow$ & $p\rho = A_i \neq A_j = q\rho$  \\
                3. & $\bigcup_{i\in I} A_i = X$ \\
                & $\bigcup_{i \in I} A_i = \bigcup_{x \in X} x\rho$  \\
                & $x\in x\rho. (x,x)\in \rho$   \\
                $\Rightarrow$ & $X = \bigcup_{x\in X}x \subseteq \bigcup_{x\in X} x\rho$    \\
                & $x\rho \subseteq X \rightarrow \bigcup_{x\in X}x\rho \subseteq X$   \\
                $\Rightarrow$ & $\bigcup_{i\in I} A_i = X$  \\
                1,2,3$\Rightarrow$ & $\Phi(\rho)$ is a partition. 
            \end{tabular}
        \end{center}

        2. $\pi$ is a partition $\to \Psi(\pi)$ is an equivalence.

        \begin{center}
            \begin{tabular}{c c}
                1. & $\forall x \in X, (x, x) \in \Psi(\pi)$ \\
                $\Rightarrow$ & $1_X \in \Psi(\pi)$ \\
                2. & $\forall x, y\in A_i \rightarrow y,x\in A_i$   \\
                $\Rightarrow$ & $(x,y)\in A_i, (y,x)\in A_i$    \\
                $\Rightarrow$ & $\forall (x,y)\in \Psi(\pi), (y,x)\in \Psi(\pi)$    \\
                3. & $\forall (x,y), (y,z) \in \Psi(\pi)$   \\
                $\Rightarrow$ & $(x,y)\in A_i, (y,z)\in A_j$   \\
                $\Rightarrow$ & $i = j \wedge (x,z) \in A_i$    \\
                $\Rightarrow$ & $(x,z) \in \Psi(\pi)$   \\
                1,2,3$\Rightarrow$ & $\Psi(\pi)$ is an equivalence.
            \end{tabular}
        \end{center}

        3. $\forall$ equivalence $\rho$ on $X \to \Psi(\Phi(\rho)) = \rho$

        \begin{center}
            \begin{tabular}{c c}
                & $\forall (x,y)\in \rho$   \\
                $\Leftrightarrow$ & $\exists p \rightarrow x,y\in p\rho$   \\
                $\Leftrightarrow$ & $\exists i\in I, x,y \in A_i, A_i \in \Phi(\rho)$    \\
                $\Leftrightarrow$ & $(x,y)\in \Psi(\Phi(\rho))$ \\
                $\Leftrightarrow$ & $\rho = \Psi(\Phi(\rho))$   \\
            \end{tabular}
        \end{center}

        4. $\forall$ partition $\pi$ of $X \to \phi(\Psi(\pi)) = \pi$.

        \textcolor{purple}{Analogue with 3 }.
    \end{proof}
    
\end{Prop}

\begin{Def}[$\rho$-classes,quotient set, natural map].

    If $\rho$ is an equivalence on $X$, we write $x \rho y $ or $x\equiv y \mod{\rho }$.
    \begin{align*}
        &\rho\text{-classes, (equivalence-classes):}\\
        &\text{  The sets } x\rho \text{ that from the partition associated with the equivalence.} \\
        &\text{quotient set :}  \\
        &\text{The set of }\rho\text{-classes, whose elements are the subsets }x\rho.
    \end{align*}
\end{Def}

\begin{Sym}[$X/\rho, \rho^{\natural}$]
    \begin{align*}
        X/\rho :&\text{ All quotient sets of X by }\rho  \\
        \rho^{\natural} :&\text{ onto map:} X\rightarrow X/\rho \\
        &\forall x\in X, x\rho^{\natural} = x\rho \text{ (is a set.)}
    \end{align*}
\end{Sym}

\begin{Prop}$\forall$ relation $\phi$ $\Rightarrow \phi \circ \phi^{-1} $ is an equivalence.

\begin{proof}
    \begin{align*}
        &\phi \circ \phi^{-1} \\
        &=\{(x,y)\in X\times X: (\exists z\in X)(x,z)\in \phi ,(y,z)\in \phi\}   \\
        &=\{(x,y)\in X\times X: x\phi =y\phi \}
    \end{align*}
    The reflexive, symmetric and transitive is clear to be proved by quotient set theory.
\end{proof}
\end{Prop}

\begin{Def}[kernel of a relations].
    \[\ker \phi =\phi \circ \phi^{-1}\]
    Notice that $\ker \rho^{\natural} =\rho $ 
    \begin{proof}
        \begin{align*}
            & \ker \rho^{\natural}  \\
            =& \rho^{\natural} \circ (\rho^{\natural})^{-1} \\
            =& \{(x,y)\in X^2:(\exists z\in X/\rho)(x,z),(y,z)\in \rho^{\natural}\}    \\
            =& \{(x,y)\in X^2:x\rho^{\natural}=y\rho^{\natural}\}   \\
            & x\rho^{\natural} = y\rho^{\natural}   \\
            \Leftrightarrow& x\rho = y\rho  \\
            \Leftrightarrow& x\rho y    \\
            \Rightarrow& \ker \rho^{\natural} = \rho
        \end{align*}
    \end{proof}
\end{Def}

\begin{Rmk}
    \begin{align*}
        &{\rho_i:i\in I} \text{ is a non-empty family of equivalences on a set } X   \\ 
        &\rightarrow \bigcap\{\rho_i:i\in I\}\text{ is again an equivalence.}
    \end{align*}
    \begin{proof}
        Firstly, we prove some laws between operators of sets and composition operator of relations.

        \begin{center}
        \begin{tabular}{c c}
            & $a,b,c$ are relations on $X$ \\
            1 & $(a\cap b)\circ c = a \circ c \cap b \circ c$    \\
            & $(x,y)\in (a \cap b)\circ c$  \\
            $\Leftrightarrow $ & $\exists z,(x,z)\in a \wedge (x,z) \in b \wedge (z,y)\in c$    \\
            $\Leftrightarrow $ & $\exists z, [(x,z)\in a \wedge (z,y)\in c] \wedge [(x,z)\in b \wedge (z,y)\in c]$  \\
            $\Leftrightarrow $ & $(x,y)\in a\circ c\cap b \circ c$  \\ 
            &   \\
            2 & $(a \cap b)^{-1}=a^{-1} \cap b^{-1}$    \\
            & $(x,y)\in (a\cap b)^{-1}$ \\
            $\Leftrightarrow $ & $(y,x) \in a \cap b$   \\
            $\Leftrightarrow $ & $(y,x) \in a \wedge (y,x) \in b$   \\
            $\Leftrightarrow $ & $(x,y) \in a^{-1} \wedge (x,y) \in b^{-1}$ \\
            $\Leftrightarrow $ & $(x,y) \in a^{-1} \cap b^{-1}$ \\
            & Particularl, this law is also true for any intersection of sets.  \\
            & $(\bigcap a_i)^{-1} = \bigcap a_i^{-1}$
        \end{tabular}
        \end{center}
        Now, Let us proof this Remark!

        \begin{center}
            \begin{tabular}{c c}
                reflaxive: & $\forall i \in I, 1_X\in \rho_i \rightarrow 1_X \subseteq \bigcap \rho_i$  \\
                symmetric: & $(\bigcap \rho_i)^{-1} = \bigcap \rho_i^{-1} = \bigcap \rho_i$ \\
                transitive: & $(\bigcap \rho_i)^2\text{\textcolor{red}{?}}\subseteq \bigcap \rho_i^2 = \bigcap \rho_i$  \\
                & $(\bigcap \rho_i)\circ (\bigcap \rho_i) ?\subseteq (\bigcap \rho_i)\circ \rho_i$   \\
                $=$ & $\bigcap \rho_i^2$    \\
                & Use the proof of elements.    \\
                & $\forall (x,y),(y,z) \in \bigcap \rho_i$  \\
                $\Rightarrow $ & $\forall i \in I, (x,y),(y,z)\in \rho_i$   \\
                $\Rightarrow $ & $\forall i \in I, (x,z) \in \rho_i$ \\
                $\Rightarrow $ & $(x,z) \in \bigcap \rho_i$
            \end{tabular}
        \end{center}

    \end{proof}
\end{Rmk}

\begin{Def}[$\mathbf{R^e}$].

    $\mathbf{R}$ is any relation on $X$, The family of equivalences containing $\mathbf{R}$ is non-empty$X^2$. Hence the intersection of all the equivalences containing $\mathbf{R}$ is an equivalence. We call it the equivalence generated by $\mathbf{R}$, denoted it by $\mathbf{R^e}$.
\end{Def}

\begin{Def}[transitive closure].

    Let $\mathbf{S}$ be a relation on $X, 1_X \subseteq \mathbf{S}$, then
    \[
        \mathbf{S}\subseteq \mathbf{S}\circ \mathbf{S}\subseteq \mathbf{S}\circ \mathbf{S}\circ \mathbf{S}\cdots 
    \]
    The relation $\mathbf{S}^{\infty}=\bigcup\{\mathbf{S}^n: n\in \mathbf{Z}^+\}$ is called the \emph{Transitive Closure} of the relation $\mathbf{S}$.
\end{Def}

\begin{Lem}\label{Lem:1.4.8}
    Every reflexive relation $\mathbf{S}$, the relation $\mathbf{S}^{\infty}$ is the smallest transitive relation on $X$ containing $\mathbf{S}$.

    \begin{proof}
        \begin{align*}
            \text{Transitive: } & \\
            &\forall (x,y),(y,z)\in \mathbf{S}^{\infty}\rightarrow  \exists m,n (x,y)\in \mathbf{S}^m \wedge (y,z)\in \mathbf{S}^n  \\
            \Rightarrow& (x,z) \in \mathbf{S}^{m+n} \subseteq \mathbf{S}^{\infty}   \\
            \text{Smallest: } & \\
            &\forall \mathbf{T} \text{ is transitive relation containing }\mathbf{S}.\\
            \Rightarrow& \mathbf{S}^2=\mathbf{S}\circ \mathbf{S}\subseteq \mathbf{T}\circ \mathbf{T}\subseteq T \\
            \Rightarrow& \forall n \in \mathbb{Z}^+, \mathbf{S}^n \subseteq T   \\
            \Rightarrow& \mathbf{S}^{\infty} \subseteq T
        \end{align*}
    \end{proof}
\end{Lem}

\begin{Prop}\label{Prop:1.4.9}
    $\mathbf{R^e}=[\mathbf{R} \cup \mathbf{R}^{-1} \cup 1_X]^{\infty}$
    \begin{proof}
        Equivalence: 
        \begin{align*}
            &\mathbf{E}=\mathbf{R} \cup \mathbf{R}^{-1} \cup 1_X]^{\infty} \text{ is transitive and contains } \mathbf{R}    \\
            \Rightarrow& 1_X \subseteq \mathbf{R}\cup \mathbf{R^{-1}}\cup 1_X \subseteq \mathbf{E}   \\
            \Rightarrow& \mathbf{E} \text{ is reflaxive.}  \\
            &\mathbf{S}=\mathbf{R} \cup \mathbf{R}^{-1} \cup 1_X \text{ is symmetric.}  \\
            \Rightarrow& \mathbf{S}^n=(\mathbf{S^{-1}})^n=(\mathbf{S^n})^{-1}   \\
            \Rightarrow& \mathbf{S}^n \text{ is symmetric}  \\
            \textcolor{red}{?}\Rightarrow& \mathbf{S}^{\infty} \text{ is symmetric}   \\
            \Rightarrow& \mathbf{E} \text{ is an equivalence relation containing }\mathbf{R}
        \end{align*}

        Smallest:
        \begin{align*}
            & \forall \text{equivalence } \sigma \wedge \mathbf{R} \subseteq \sigma   \\
            \Rightarrow& 1_X \subseteq \sigma \wedge \sigma^{-1} = \sigma   \\
            \Rightarrow& \mathbf{S}=\mathbf{R} \cup \mathbf{R}^{-1} \cup 1_X \subseteq \sigma \\
            \Rightarrow& \mathbf{S}\circ \mathbf{S} \subseteq \sigma \circ \sigma =\sigma \\
            \Rightarrow& \forall n\in \mathbb{Z}^+, \mathbf{S}^n \subseteq \sigma   \\
            \Rightarrow& \mathbf{S}^{\infty} \subseteq \sigma 
        \end{align*}
        The foregoing $\Rightarrow \mathbf{E}$ is the smallest equivalence on $X$ containing $\mathbf{R}$.
        \[\text{Def}\Rightarrow \mathbf{R^e}=[R\cup R^{-1}\cup 1_X]^{\infty}\]
    \end{proof}
\end{Prop}

\begin{Prop}\label{Prop:1.4.10}
    \begin{align*}
        (x,y)\in \mathbf{R^e} \Leftrightarrow \text{either } x=y    \\
        \text{or } x=z_1\rightarrow z_2\rightarrow \cdots\rightarrow z_n\rightarrow y
    \end{align*}
    in witch,$\forall i \in \{1,2,\cdot,n-1\},$ either $(z_i,z_{i+1})\in \mathbf{R}$ or $(z_{i+1},z_i)\in \mathbf{R}$

    \textcolor{purple}{\textbf{Remark:} $\forall (x,y)\in \mathbf{R}^e$, either $x=y$ or $(x,y)$ is added by the transitive closure.}
\end{Prop}

\subsection[5]{Congruences}

\begin{Def}[Compatible]
    Let $S$ be a semigroup. A relation $\mathbf{R}$ on the set $S$ is called:
    \begin{center}
        \begin{tabular}{c|c}
            Left Compatible & $(\forall s,t,a \in S)(s,t)\in \mathbf{R} \Rightarrow (as, at)\in \mathbf{R}$   \\
            Right Compatible & $(\forall s,t,a \in S)(s,t)\in \mathbf{R} \Rightarrow (sa,ta)\in \mathbf{R}$   \\
            Compatible & $(\forall s,t,s',t' \in S)[(s,t)\in\mathbf{R}\wedge (s',t')\in \mathbf{R}] \Rightarrow (ss',tt')\in \mathbf{R}$
        \end{tabular}
    \end{center}
\end{Def}

\begin{Def}[Congruence].
    \begin{center}
        \begin{tabular}{c|c}
            Left Congruence & Left compatible equivalence   \\
            Right Congruence & Right compatible equivalence \\
            Congruence  & Compatible equivalence    \\
        \end{tabular}
    \end{center}
\end{Def}

\begin{Prop}\label{Prop:1.5.1}
    A relation $\rho$ on a semigroup $S$ is a  \emph{Congruence} $\Leftrightarrow$
    $\rho$ is both a \emph{Left Congruence} and a \emph{Right Congruence}.
    \begin{proof}.
        \begin{center}
            \begin{tabular}{c c}
                & $\rho$ is congruence $\Rightarrow $ $\rho$ is both left and right congruence. \\
                & $\forall a\in S, (a,a) \in \rho$  \\
                $\Rightarrow $ & $(as,at)\in \rho \wedge (sa, ta)\in \rho$  \\
                $\Rightarrow $ & $\rho$ is both left and right congruence.  \\
                &   \\
                & $\rho$ is both left and right congruence $\Rightarrow \rho$ is congruence.    \\
                & $\forall (s,t),(s',t')\in \mathbf{R}$  \\
                $\Rightarrow $ & $(ss',ts')\in R \wedge (ts',tt') \in \mathbf{R}$    \\
                $\Rightarrow $ & $(ss',tt')\in \mathbf{R}$
            \end{tabular}
        \end{center}
    \end{proof}
\end{Prop}

If $\rho$ is a congruence on a semigroup $S$ then we can define a binary operation on the quotient set $S/\rho$ in a natural way as follows:
\begin{equation}\label{Eq_1_5_1}
    (a\rho)(b\rho)=(ab)\rho
\end{equation}

This is well-defined precisely because $\rho$ is compatible: 
$\forall a,a',b,b' \in S$
\begin{align*}
    & a\rho=a'\rho \wedge b\rho  =b'\rho                \\
    \Rightarrow & (a,a')\in \rho \wedge (b,b') \in \rho \\
    \Rightarrow & (ab,a'b') \in \rho                    \\
    \Rightarrow & (ab)\rho = (a'b')\rho                 \\
\end{align*}
this operation is easily seen to be associatice, and so $S/\rho$ is a semigroup.

\begin{Them}\label{Them:1.5.2}
    Let $S$ be a semigroup, and let $\rho$ be a congruence on $S$. Then $S/\rho$ is a semigroup with respect to the operation defined by period, and the map $\rho^{\natural}$ from $S$ onto $S/\rho$ given by $x\rho^{\natural}=x\rho$ is a morphism.

    Let $T$ be a semigroup and let $\phi:S\rightarrow T$ be a morphism. Then the relation
    \[
        \ker \phi = \phi \circ \phi^{-1}=\{(a,b)\in S\times S| a\phi = b\phi\}
    \]
    is a congruence on $S$, and there is a monomorphism $\alpha:S/\ker \phi \rightarrow T$ such that $\Image \alpha = \Image \phi$ and the diagram commutes.
    \[
        \begin{tikzcd}
            S
            \arrow[dd, "(\ker \phi)^{\natural}"']
            \arrow[r,"\phi"] &
            T   \\
            \\
            S/\ker \phi
            \arrow[uur, "\alpha"'] &
        \end{tikzcd}
    \]

    \textbf{Remark:} $S/\ker \phi \rightarrow T$ is a one-one monomorphism, so that $S \rightarrow S/\ker \phi$ do not lost the main infomation of $\phi: S\rightarrow T$.
    \begin{proof}
        It is easy to verify that $\rho^{\natural}$ is a morphism.
        \begin{center}
            \begin{tabular}{c c}
                & $(xy)\rho^{\natural}$ \\
                $=$ & $(xy)\rho$        \\
                $\ref{Eq_1_5_1}=$ & $(x\rho)(y\rho)$  \\
                $=$ & $(x\rho^{\natural})(y\rho^{\natural})$
            \end{tabular}
        \end{center}

        The $\ker \phi = \phi \circ \phi^{-1}$ is a congruence.
        \begin{center}
            \begin{tabular}{c c}
                & $\forall (a,a'),(b,b')\in \ker \phi$  \\
                $\Rightarrow $ & $a\phi = a'\phi \wedge b\phi=b'\phi$   \\
                $\Rightarrow $ & $(ab)\phi=(a\phi)(b\phi)=(a'\phi)(b'\phi)=(a'b')\phi$      \\
                $\Rightarrow $ & $(ab, a'b')\in \ker\phi$
            \end{tabular}
        \end{center}

        Let $\kappa:= \ker \phi$. Define $\alpha:S/\kappa \rightarrow T$
        \[
            (\forall a \in S)(a\kappa)\alpha = a\phi
        \]
        Then $\alpha$ is well-defined and one-one, since
        \[
            a\kappa =b\kappa \Leftrightarrow (a,b)\in \kappa \textcolor{blue}{\Leftrightarrow} a\phi = b\phi \Leftrightarrow (a\kappa)\alpha = (b\kappa)\alpha
        \]
        It is also a morphism, since
        \begin{align*}
            & \forall a,b \in S \\
            & [(a\kappa)(b\kappa)]\alpha    \\
            =& [(ab)\kappa]\alpha   \\
            =& (ab)\phi \\
            =& (a\phi)(b\phi)   \\
            =& [(a\kappa)\alpha][(b\kappa)\alpha]   \\
        \end{align*}

        Clearly $\Image \alpha = \Image \phi$, and from the definition of $\alpha$ it is clear that:
        \[
            (\forall a \in S) a\kappa^{\natural}\alpha = a\phi
        \]
    \end{proof}
\end{Them}

\begin{Them}\label{Them:1.5.3}
    Let $\rho$ be a congruence on a semigroup $S$, and let $\phi: S \to T$ be a morphism such that $\rho \subseteq \ker \phi$. Then there is a unique morphism $\beta:S/\rho \to T$ such that $\Image \beta = \Image \phi$ and such that the diagram commutes.
    \[\begin{tikzcd}
        S        
        \arrow[dd, "\rho^{\natural}"'] 
        \arrow[r,"\phi"] &
        T   \\
        \\
        S/\rho
        \arrow[uur, "\beta"']
    \end{tikzcd}\]

    \textbf{Remark:} $S/\rho \rightarrow T$ is a morphism but not always one-one, some infomation lost by $\rho$ and some by $\beta$. Both of the losses constitute the $\phi$'s.
    \begin{proof}
        Define $\beta: S/\rho\rightarrow T$:
        \[
            (a \rho) \beta = a\phi (a \in S)
        \]

        $\beta$ is well-defined and not always one-one, since, $\forall a,b \in S$
        \[
            a\rho = b \rho \Leftrightarrow (a,b) \in \rho \textcolor{blue}{\Rightarrow} (a,b) \in \ker \phi \Leftrightarrow a \phi = b \phi \Leftrightarrow (a\rho)\beta = (b\rho)\beta
        \]

        $\beta$ is a morphism. 

        \begin{center}
        \begin{tabular}{c c}
            & $\forall a,b\in S. [(a\rho)(b\rho)]\beta$   \\
            $=$ & $[(a b)\rho]\beta$    \\
            $=$ & $(a b)\phi$   \\
            $=$ & $(a)\phi (b)\phi$ \\
            $=$ & $[(a\rho)\beta][(b\rho)\beta]$
        \end{tabular}
        \end{center}
        
        $\rho^{\natural} \circ \beta = \phi$
        
        \begin{center}
            \begin{tabular}{c c}
                & $\forall a \in S, a \rho^{\natural} \circ \beta$ \\
                $=$ & $(a\rho)\beta$    \\
                $=$ & $a\phi$
            \end{tabular}
        \end{center}

        so that: $\Dom \phi = \Dom \rho^{\natural} \circ \beta = S \Rightarrow \Image \phi = \Image \rho^{\natural} \circ \beta$

        $\beta$ is unique.
        \begin{center}
            \begin{tabular}{c c}
                & $\forall \gamma \in (S/\rho \rightarrow T) \wedge (a \rho)\gamma = a \phi  $ \\
                $=$ & $(a\rho)\beta$    \\
                $\wedge$ & $\Dom \gamma = \Dom \beta$   \\
                $\Rightarrow $ & $\gamma = \beta$
            \end{tabular}
        \end{center}
        
        Any morphism satisfying $\rho^{\natural} \circ \beta = \phi$ must be defined by the rule.
    \end{proof}
\end{Them}

    Theorem \ref{Them:1.5.3} has a application to the situation where $\rho$ and $\sigma$ are congruences on the $S$ and where $\rho \subseteq \sigma$. The theorem implies that there is a morphism $\beta$ from $S/\rho$ onto $S/\sigma$ such that the diagram commutes.
    \textcolor{blue}{
    \[
        \rho \subseteq \ker \sigma^{\natural} = \sigma
    \]}
\[\begin{tikzcd}
    S
    \arrow[dd, "\rho^{\natural}"']
    \arrow[r, "\sigma^{\natural}"] &
    S/\sigma    \\
    \\
    S/\rho
    \arrow[uur, "\beta"']
\end{tikzcd}\]

The morphism $\beta$ is given by
\[
    \forall (a\in S)(a\rho)\beta = a\sigma
\]

and the congruence $\ker \beta$ on $S/\rho$ is given by
\[
    \ker \beta = 
    \{  
        (a\rho,b\rho)\in (S/\rho)^2 | a\rho\beta = b\rho\beta \Leftrightarrow  a\sigma=b\sigma \Leftrightarrow (a,b)\in \ker\sigma = \sigma 
    \}
\]

It is usual to write $\ker \beta$ as $\sigma/\rho(\frac{S/\rho}{S/\sigma}\approx\frac{\sigma}{\rho})$. From Theorem\ref{Them:1.5.2} it now follows that there is an isomorphism $\alpha:(S/\rho)/(\sigma/\rho) \to S/\sigma$ defined by:
\[
    (\forall a\in S)[(a\rho)(\sigma/\rho)]\alpha = a\sigma
\]
\textbf{Remark:}
This map $\alpha$ may be a functor, defined by:
\begin{align*}
    a:\cate{Set} &\to \cate{Set} \\
    a(\{A\}) &= A  \\
    \alpha &:= a^n
\end{align*}

\begin{Them}\label{Them:1.5.4}
    Let $\rho, \sigma$ be congruences on a semigroup $S$ such that $\rho \subseteq \sigma$. Then:
    \[
        \sigma/\rho = 
        \{
            (x\rho, y\rho)\in S/\rho \times S/\rho | (x,y)\in \sigma
        \}
    \]
    is a congruence on $S/\rho$, and $(S/\rho)/(\sigma/\rho) \simeq S/\sigma$. The dragram commutes.
    \[\begin{tikzcd}
        S
        \arrow[r, "\sigma^{\natural}"]
        \arrow[dd, "\rho^{\natural}"'] &
        S/\sigma    \\
        \\
        S/\rho
        \arrow[uur, "\beta"]
        \arrow[r, "(\sigma/\rho)^{\natural}"'] &
        (S/\rho)/(\sigma/\rho)
        \arrow[uu, "\alpha"']
    \end{tikzcd}\]
    \begin{proof}
        \begin{align*}
            & \rho \subseteq \ker \sigma^{\natural} = \sigma \\
            \ref{Them:1.5.3}\Rightarrow& \sigma^{\natural} = \rho^{\natural}\circ \beta \\
            & \ker \beta = (\sigma/\rho)    \\
            \Rightarrow& \ker \beta^{\natural} = \ker(\sigma/\rho)^{\natural} = \sigma/\rho \\
            \ref{Them:1.5.2}\Rightarrow& \beta = (\sigma/\rho)^{\natural}\circ \alpha \wedge \alpha \text{ is a Isomorphism.}
        \end{align*}
    \end{proof}
\end{Them}

The intersection of a non-empty family of congruences on a semigroup $S$ is also a congruence on $S$.

\begin{align*}
    & \forall (s,t),(s',t')\in R = \bigcap_i R_i,  \\
    \Rightarrow & (s, t),(s',t') \in R_i    \\
    \Rightarrow & (ss', tt'),(s's, t't) \in R_i \\
    \Rightarrow & (ss', tt'),(s's, t't) \in R
\end{align*}

\begin{Def}$\mathbf{R}^{\sharp}$
    \[
        \mathbf{R}^{\sharp} = \{\bigcap \mathbf{C} : \mathbf{R} \subseteq \mathbf{C} \wedge \mathbf{C}\text{ is a congruence}\}
    \]
    $\mathbf{R}^{\sharp}$ is unique smallest congruence on $S$ containing $\mathbf{R}$.
    Because of the analogue of equivalence, the $\mathbf{R}^{\sharp}$ is unique and smallest. \textcolor{purple}{?maybe it is not true, but the proof may be easy.}
\end{Def}

To describe the $\mathbf{R}^{\sharp}$, we firstly describe the $\mathbf{R}^{c}$.

\begin{Def}$\mathbf{R}^{c}$
    
    $\mathbf{R}$ is arbitrary relation on $S$.
    \[
        \mathbf{R}^{c} =
        \{
            (xay, xby): x,y \in S^1, (a,b)\in \mathbf{R}
        \}
    \]
\end{Def}

\begin{Lem}\label{Lem:1.5.5}
    $\mathbf{R}^c$ is the smallest left and right compatible relation containing $\mathbf{R}$.
    \begin{proof}
         \begin{align*}
            \mathbf{R}\subseteq \mathbf{R}^c \text{: } & 1 \in S^1    \\
            \Rightarrow & \forall (a,b) \in \mathbf{R}, (1a1,1b1) \in \mathbf{R}^c  \\
            \Rightarrow & \mathbf{R} \subseteq \mathbf{R}^c \\
            & \\
            \text{Compatible: } & \forall (u,v) \in \mathbf{R}^c,\forall w \in S  \\
            \Rightarrow & \exists (a,b)\in \mathbf{R}\rightarrow u=xay, v=xby   \\
            \Rightarrow & wu = wxay = (wx)ay, wv = wxby = (wx)by  \\
            \Rightarrow & (wu, wv) \in \mathbf{R}^c \\
            \Rightarrow & \mathbf{R}^c \text{ is left compatible.}  \\
            & \text{The right compatible is an analogue.}   \\
            & \\
            \text{Smallest: } & \forall \mathbf{S} \text{ satisfying left and right compatible and containing } \mathbf{R}   \\
            \Rightarrow & \forall x,y \in S^1, \forall (a,b) \in \mathbf{R} \\
            \Rightarrow & (xay, xby) \in \mathbf{S} \\
            \Rightarrow & \mathbf{R}^c \subseteq \mathbf{S}
         \end{align*}
    \end{proof}
\end{Lem}

\begin{Lem}\label{Lem:1.5.6}
    Let $\mathbf{R}, \mathbf{S}$ be relations on a semigroup $S$. Then:
    \begin{enumerate}
        \item $\mathbf{R} \subseteq \mathbf{S} \Rightarrow \mathbf{R}^c \subseteq \mathbf{S}^c$ 
        \item $(\mathbf{R}^{-1})^c = (\mathbf{R}^c)^{-1}$
        \item $(\mathbf{R} \cup \mathbf{S})^c = \mathbf{R}^c \cup \mathbf{S}^c$
        \item $1_S^c=1_S$
    \end{enumerate}
    \begin{proof}
        \begin{align*}
            1. & \forall (xuy, xvy)\in \mathbf{R}^c \\
            \Rightarrow& (u,v) \in \mathbf{R} \subseteq \mathbf{S}  \\
            \Rightarrow& (xuy, xvy) \in \mathbf{S}^c \\
            & \\
            2. & \forall (xuy, xvy) \in (\mathbf{R}^{-1})^c \\
            \Rightarrow& (u,v) \in \mathbf{R}^{-1}  \\
            \Rightarrow& (v,u) \in \mathbf{R}   \\
            \Rightarrow& (xvy, xuy) \in \mathbf{R}^c    \\
            \Rightarrow& (xuy, xvy) \in (\mathbf{R}^c)^{-1} \\
            & \\
            3. & \forall (xuy,xvy) \in (\mathbf{R}\cup \mathbf{S})^c    \\
            \Rightarrow& (u,v)\in \mathbf{R} \vee (u,v) \in \mathbf{S} \\
            \Rightarrow& (xuy, xvy) \in \mathbf{R}^c \vee (xuy, xvy)\in \mathbf{S}^c \\
            \Rightarrow& (xuy, xvy) \in \mathbf{R}^c \cup \mathbf{S}^c \\
            & \\
            &
            \begin{cases}
                1_S \subseteq 1_S^c \\
                \forall (xay,xay) \in 1_S^c\rightarrow (xay, xay)\in 1_S \rightarrow 1_S^c \subseteq 1_S
            \end{cases}         \\
            \Rightarrow & 1_S^c = 1_S       \\
        \end{align*}
    \end{proof}
\end{Lem}

\begin{Lem}\label{Lem:1.5.7}
    Let $\mathbf{R}$ be a left and right compatible relation on a semigroup $S$. Then: $\mathbf{R}^n$ is left and right compatible for every $n\geq 1$.
    \textbf{Formula: }
    \begin{align*}
        & \forall \mathbf{R} \text{ is a left and right compatible relation on a semigroup }S \\
        \Rightarrow &
        \forall n\in \mathbb{N}^+, \mathbf{R}^n \text{ is also left and right compatible.}
    \end{align*}

    \begin{proof}
        \begin{align*}
            & \forall (s,t)\in \mathbf{R}^n     \\
            \Rightarrow & \exists z_1,z_2,\cdots, z_{n-1}\rightarrow (s,z_1),(s,z_2),\cdots, (z_n-1,t) \in \mathbf{R}    \\
            \Rightarrow &\forall a \in S \rightarrow 
            \begin{cases}
                (as,az_1),(az_1,az_2),\cdots, (az_{n-1},a t) \in \mathbf{R}  \\
                (sa,z_1a),(z_1a,z_2,a),\cdots, (z_{n-1}a,ta) \in \mathbf{R}  \\ 
            \end{cases} \\
            \Rightarrow & (as,at),(sa,ta) \in \mathbf{R}
        \end{align*}
    \end{proof}
\end{Lem}

\begin{Prop}\label{Prop:1.5.8}
    $\mathbf{R}^{\sharp} = (\mathbf{R}^c)^e$.

    \begin{proof}
        \begin{align*}
            \text{Congruence:}\\
            \ref{Prop:1.4.9}\Rightarrow &
            \begin{cases}
                (\mathbf{R}^c)^e \text{ is an equivalence.} \\
                \mathbf{R}^c \subseteq (\mathbf{R}^c)^e     \\
                \rightarrow \mathbf{R} \subseteq (\mathbf{R}^c)^e       \\
            \end{cases}             \\
            \text{Let:} & \mathbf{S} = \mathbf{R}^c \cup (\mathbf{R}^c)^{-1} \cup 1_S \\
            \ref{Lem:1.5.6}=& \mathbf{R}^c \cup (\mathbf{R}^{-1})^c \cup 1_S^c \\
            \ref{Lem:1.5.6}=& (\mathbf{R} \cup \mathbf{R^{-1}} \cup 1_S )^c    \\
            \ref{Lem:1.5.5}\Rightarrow & \mathbf{S} \text{ is left and right compatible.}   \\
            \ref{Lem:1.5.7} \Rightarrow & \mathbf{S}^n \text{ is left and right compatible.}    \\
            \ref{Prop:1.5.1}\Rightarrow & (\mathbf{R}^c)^e \text{ is a congruence on } S \text{ containing } \mathbf{R}.    \\
            & \forall (s,t) \in (\mathbf{R}^c)^e, \forall a \in S   \\
            \Rightarrow& (as, at) \in \mathbf{S}^n \subseteq (\mathbf{R}^c)^e \wedge (sa, ta) \in \mathbf{S}^n \subseteq (\mathbf{R}^c)^e  \\
            \\
            \text{Smallest:} & \forall \text{congruence } \kappa \wedge \mathbf{R} \subseteq \kappa \\
            & \ref{Lem:1.5.5}\Rightarrow \kappa^c = \kappa  \\
            \Rightarrow & \mathbf{R}^c \subseteq \kappa^c = \kappa  \\
            & \kappa \text{ is an equivalence}  \\
            \Rightarrow & (\mathbf{R}^c)^e \subseteq \kappa
        \end{align*}
    \end{proof}
    \textbf{Remark: }The method used to prove this smallest property is  describing the smallest equivalence condition could somehow prove the smallest congurence.
\end{Prop}

\begin{Def}[connected by an elementary $\mathbf{R}$-transition].

    By analogue with Proposition \ref{Prop:1.4.9} and \ref{Prop:1.4.10} we may write the last result in more elementary terms. First, if $c, d \in S$ are such that
    \[
        c = xay, d = xby.
    \]
for some $x, y$ in $S^1$, where either $(a, b)$ or $(b, a)$ belongs to $\mathbf{R}$, we say that $c$ is connected to $d$ by an \emph{elementary $\mathbf{R}$-transition}.
\end{Def}

\begin{Prop}\label{Prop:1.5.9}
    Let $\mathbf{R}$ be a relation on a semigroup $S$, and let $a, b \in S$. Then $(a, b)\in \mathbf{R}^{\sharp} \Leftrightarrow $either $a=b$ or, for some $n \in N$, there is a sequence
    \[
        a = z_1 \rightarrow z_2 \rightarrow \cdots \rightarrow z_n = b
    \]
    of elementary $\mathbf{R}$-transitions connecting $a$ to $b$.
    \begin{proof}
        \textcolor{red}{This proof may be wrong?}
        \begin{align*}
            & \forall (a, b) \in \mathbf{R}^{\sharp} \wedge (a, b) \notin \mathbf{R}  \\
            \Leftrightarrow& \exists n, (a, b) \in ((\mathbf{R} \cup \mathbf{R}^{-1} \cup 1_S)^c)^n   \\
            \Leftrightarrow& (a, x_1) \in \mathbf{S}^c, (x_1, x_2) \in \mathbf{S}^c, \cdots, (x_{n-1}, b) \in \mathbf{S}^c   \\
            \Leftrightarrow& \exists p_i, q_i \in S^1, \exists (x_{il}, y_{il}) \in \mathbf{R}  \\
            \rightarrow&  x_i = p_i x_{il} q_i; y_{i} = p_i y_{il} q_i \vee
            x_i = p_i y_{il} q_i; y_{i} = p_i x_{il} q_i    \\
            \Leftrightarrow& a = \prod_{0}^{n} p_i \cdot x_{nl} \cdot \prod_{n}^{0} q_i; b = \prod_{0}^{n} p_i \cdot x_{nl} \cdot \prod_{n}^{0} q_i \\
            \Leftrightarrow& x \text{ is connected to }y \text{ by the elementary } \mathbf{R} \text{-transitions.}
        \end{align*}
    \end{proof}
\end{Prop}

\begin{Def}[$\mathbf{E}^{\flat}$]
    $\mathbf{E}$ is an arbitrary equivalence on a semigroup $S$.
    \[
    \mathbf{E}^{\flat}=
    \{
        (a,b) \in S \times S: (\forall x, y \in S^1)(xay,xby) \in \mathbf{E}
    \}
\]
\end{Def}

\begin{Prop}\label{Prop:1.5.10}
    Let $\mathbf{E}$ be an equivalence on a semigroup $S$. Then $\mathbf{E}^\flat$ is the \emph{lagrest congruence} on $S$ contained in $\mathbf{E}$.
    \begin{proof}.
        \begin{center}
            \begin{tabular}{c c}
                & $\mathbf{E}^{\flat}$ is an equivalence.   \\
                $\Leftarrow$ & 
                $\begin{cases}
                    1_S \subseteq \mathbf{E}^{\flat}  \\
                    \mathbf{E} = \mathbf{E}^{-1} \Rightarrow (\mathbf{E}^{\flat})^{-1} = \mathbf{E}^{\flat} & \text{by definition.} \\
                    \textcolor{red}{\text{Transitive ?}} \\
                \end{cases}$    \\  \\
                & $\mathbf{E}^{\flat}$ is compatible.     \\
                $\Leftarrow$ & $\forall (xay, xby) \in \mathbf{E}^{\flat}, \forall c \in S$ \\
                $\Rightarrow $ & $(cxay, cxby) = ((cx)ay, (cx)by) \in \mathbf{E}^{\flat}$     \\
                analogue$\Rightarrow $ & $(xayc, xbyc) \in \mathbf{E}^{\flat}$  \\
                \\
                & $\forall (a, b)\in \mathbf{E}^{\flat} \Rightarrow (a, b) = (1a1,1b1) \rightarrow (a, b) \in \mathbf{E}$  \\
                $\Rightarrow $ & $\mathbf{E}^{\flat} \subseteq \mathbf{E}$ \\ \\
                & $\mathbf{E}^{\flat}$ is largest.  \\
                $\Leftarrow$ & $\forall$ congurence $\eta \subseteq \mathbf{E}$  \\
                $\Rightarrow $ & $\forall (a, b) \in \eta$  \\
                $\Rightarrow $ & $(\forall x,y \in S^1)(xay, xby) \in \eta$ \\
                $\Rightarrow $ & $(\forall x,y \in S^1)(xay, xby) \in \mathbf{E}$   \\
                $\Rightarrow $ & $\eta \subseteq \mathbf{E}$
            \end{tabular}
        \end{center}
    \end{proof}
\end{Prop}

\begin{Rmk}
    $\mathbf{R}^{\sharp}$ is defined for an arbitrary relation $\mathbf{R}$, the congruence $\mathbf{E}^{\flat}$ is defined only for an equivalence $\mathbf{E}$.

    $\mathbf{R}^{\sharp}$ is minimum congruence containing $\mathbf{R}$; $\mathbf{E}^{\flat}$ is maximum congruence contained by $\mathbf{E}$
\end{Rmk}

\begin{Def}[Syntactic Congruence]
    $\forall A \subseteq S, \exists \text{ equivalence } \pi_A$ whose classes are $A, S\backslash A$, The congruence \emph{Syntactic Congruence} of $A$ defined by:
    \[\pi_A^{\flat} = \{(u,v)\in S \times S : (\forall x,y\in S^1)[xuy \in A \Leftrightarrow xvy \in A]\}\]

    This definition is important in the theorem of automata and languages.
\end{Def}

\begin{Sym}.
    \begin{center}
        \begin{tabular}{c c}
            $\mathcal{E}(S)$ & := $\{\rho \in \mathcal{B}(S) : \rho \text{ is a equivalence}\}$    \\
            $\mathcal{C}(S)$ & := $\{\rho \in \mathcal{B}(S) : \rho \text{ is a congruence}\}$    \\ 
        \end{tabular}
    \end{center}
    Both of them are partially ordered by inclusion $\subseteq$, moreover, both are \emph{lattice}.

    \begin{center}
        \begin{tabular}{c c}
            & $\forall \rho, \sigma \in \mathcal{E}(S) $    \\
            $\Rightarrow $ & $\rho \cap \sigma \in \mathcal{E}(S)$  is their greatest lower bound.    \\
            $\Rightarrow $ & $(\rho \cup \sigma)^e \in \mathcal{E}(S)$ is their least upper bound.        \\ 
            & $\forall \rho, \sigma \in \mathcal{C}(S) $    \\
            $\Rightarrow $ & $\rho \cap \sigma \in \mathcal{C}(S) $ is their greatest lower bound.      \\
            $\Rightarrow $ & $(\rho \cup \sigma)^{\sharp} \in \mathcal{C}(S)$ is their least upper bound. \\
        \end{tabular}
    \end{center}

    \begin{proof}
        \begin{align*}
            & \rho, \sigma \in \mathcal{E}(S)   \\
            & 1_S \subseteq \rho \wedge 1_S \subseteq \sigma \Rightarrow 1_S    \subseteq \rho \cap \sigma \\
            & \rho^{-1}=\rho \wedge \sigma^{-1} = \sigma \Rightarrow \rho \cap \sigma = (\rho \cap \sigma)^{-1} \\
            & (x,y),(y,z) \in \rho \cap \sigma \Rightarrow (x,z) \in \rho \cap \sigma \\
            \Rightarrow& \rho \cap \sigma \in \mathcal{E}(S)    \\
            & \forall \gamma \subseteq \rho \wedge \gamma \subseteq \sigma \wedge \gamma \in \mathcal{E}(S) \\
            \Rightarrow& \rho \subseteq \rho \cap \sigma \in \mathcal{E}(S) \\
            \Rightarrow& \rho \cap \sigma \text{ is their greatest lower bound.}    \\
            & \\
            & \text{others are easy.}
        \end{align*}
    \end{proof}

    \begin{align*}
        & \forall \rho, \sigma \in \mathcal{C}(S)   \\
        \ref{Lem:1.5.6} \Rightarrow& (\rho \cup \sigma)^c = \rho^c \cup \sigma^c = \rho \cup \sigma      \\
        \ref{Prop:1.5.8} \Rightarrow& (\rho \cup \sigma)^{\sharp} = (\rho \cup \sigma)^e
    \end{align*}

    So $\rho \vee \sigma$ is an unambiguous definition of the join of $\rho$ and $\sigma$ either in $\mathcal{E}(S)$ or in $\mathcal{C}(S)$.

    The set $\{\rho, \sigma\}$ could be replaced by arbitrary family of equivalences or congurences. So $\mathcal{E}(S), \mathcal{C}(S)$ are \emph{complete lattice}. Both lattices have maximum element $S \times S$ and minimum element $1_S$
\end{Sym}

\begin{Prop}\label{Prop:1.5.11}
    Let $\rho, \sigma$ be equivalences on a set $S$ [congruences on a semigroup $S$]. Then $(a, b)\in \rho \bigvee \sigma \Leftrightarrow \exists n\in \mathbb{N}, \exists x_1, x_2, \cdots, x_{2n-1} \in S$ such that:
    \[
        (a, x_1)\in \rho, (x_1, x_2)\in \sigma, (x_2, x_3)\in \rho, (x_3, x_4)\in \sigma, \cdots, (x_{2n-1}, b) \in \sigma
    \]
    \textbf{Remark:}
    This Prop is a specialization of Prop\ref{Prop:1.4.10} and \ref{Prop:1.5.9} to the case where the relation $\mathbf{R}$ is the union of two equivalences or congruences is worth stating separately.

    \begin{proof}
        The result says effectively that
        \[
            \rho \bigvee \sigma = (\rho \cup \sigma)^e = (\rho \circ \sigma)^{\infty}
        \]
        Define: $\mathbf{R} = (\rho \cup \sigma)\cup (\rho \cup \sigma)^{-1}\cup 1_S = \rho \cup \sigma$.
        this Prop could convert to:
        \[\rho \bigvee \sigma = \mathbf{R}^{\infty}\]
        \begin{center}
            \begin{tabular}{c c}
                & $(\rho \subseteq \rho \cup \sigma) \wedge (\sigma \subseteq \rho \cup \sigma)$  \\
                $\Rightarrow$ & $\rho \circ \sigma \subseteq (\rho \cup \sigma)^2$  \\
                $\Rightarrow$ & $\forall n \in \mathbb{N}^+, (\rho \circ \sigma)^n \subseteq (\rho \cup \sigma)^{2n}$   \\
                $\Rightarrow$ & $(\rho \circ \sigma)^{\infty} \subseteq (\rho \cup \sigma)^{\infty} = \rho \bigvee \sigma$  \\
                & Conversely    \\
                & $1_S \subseteq \rho \wedge 1_S \subseteq \sigma$  \\
                $\Rightarrow$ & $(\rho \subseteq \rho \circ \sigma) \wedge (\sigma \subseteq \rho \circ \sigma)$    \\
                $\Rightarrow$ & $\rho \cup \sigma \subseteq \rho \circ \sigma$ \\
                $\Rightarrow$ & $\rho \bigvee \sigma = (\rho \cup \sigma)^{\infty} \subseteq (\rho \circ \sigma)^{\infty}$  \\
                & Totally   \\
                $\Rightarrow$ & $\rho \bigvee \sigma = (\rho \circ \sigma)^{\infty}$
            \end{tabular}
        \end{center}
    \end{proof}
\end{Prop}

\begin{Coly}\label{Coly:1.5.12}
    Let $\rho, \sigma$ be equivalence on a set $S$ [congruence on a semigroup $S$] such that $\rho \circ \sigma = \sigma \circ \rho$, Then $\rho \vee \sigma =  \rho \circ \sigma$.
    
    \begin{center}
        \begin{tabular}{c c}
            & $\rho, \sigma$ are equivalences on set $S$ \\
            & $(\rho \circ \sigma = \sigma \circ \rho)$ \\
            $\Rightarrow $ & $\rho \bigvee \sigma = \rho \circ \sigma$  \\
            & $\rho, \sigma$ are congruences on semigroup $S$ \\
            & $(\rho \circ \sigma = \sigma \circ \rho)$ \\
            $\Rightarrow $ & $\rho \bigvee \sigma = \rho \circ \sigma$  \\
        \end{tabular}
    \end{center}

    \begin{proof}.
        \begin{center}
            \begin{tabular}{c c}
                & $\rho \circ \sigma = \sigma \circ \rho$   \\
                & $(\rho \circ \sigma)^2$    \\
                $=$ & $\rho \circ (\sigma \circ \rho) \circ \sigma$ \\
                $=$ & $\rho \circ (\rho \circ \sigma) \circ \sigma$ \\
                $=$ & $(\rho \circ \rho) \circ (\sigma \circ \sigma)$   \\
                $=$ & $\rho \circ \sigma$   \\
                $\Rightarrow$ & $\forall n\in \mathbb{N}^{+}(\rho \circ \sigma)^n = \rho \circ \sigma$    \\
                $\Rightarrow$ & $(\rho \circ \sigma)^{\infty} = \rho \circ \sigma$  \\
                $\ref{Prop:1.5.11}$ & $\rho \bigvee \sigma = (\rho \circ \sigma)^{\infty}$   \\
                $=$ & $\rho \circ \sigma$   \\
            \end{tabular}
        \end{center}
    \end{proof}
\end{Coly}



\subsection[6]{Free Semigroups and Monoids; Presentations}
\subsection[7]{Ideals and Rees Congruences}
\subsection[8]{Lattices of Equivalences and Congruences}
\subsection[9]{Exercises}
1. An element $e$ is a semigroup $S$ is called a \emph{left identity} if $\forall x \in S\rightarrow ex = x$. Analogue is for \emph{right identity}.
An element $z$ of $S$ is called a \emph{left zero} if $\forall x \in S, zx=z$. Analogue is for \emph{right zero}.

Proof: If $S$ has a left identity $e$ and a right identity $f$, then $e=f$ and $e$ is the unique two-sided identity for $S$.
\begin{Prof}
    \begin{align*}
        e=ef=f \Rightarrow e=f  \\
        \forall e' \neq e, e' = e'e = e \Rightarrow e'=e
    \end{align*}
\end{Prof}
That semigroup has both left identity(ies) and right identity(ies) has only one identity.


Proof: If $S$ has a left zero $z$ and a right zero $u$, then $z=u$, and $z$ is the unique two-sided zero for $S$.
\begin{Prof}
    \begin{align*}
        z=zu=u \Rightarrow z=u  \\
        \forall z'\neq z, z' = z'z = z \Rightarrow z'=z
    \end{align*}
\end{Prof}

Proof: Give an example of a semigroup having two (at least) left identities and two (at least) right zeros.
\begin{align*}
    & \{e_1, e_2\}    \\
    \text{Multiplication: } &
    \begin{cases}
        e_1 e_2 = e_2   \\
        e_2 e_1 = e_1   \\
        e_1 e_1 = e_1   \\
        e_2 e_2 = e_2
    \end{cases} \\
    \text{Association: } &
    \begin{cases}
        e^3_1=e_1   \\
        e^3_2=e_2   \\
        (e_1 e_1)e_2 = e_1 e_2 = e_2 \quad e_1 (e_1 e_2) = e_1 e_2 = e_2   \\
        (e_1 e_2)e_1 = e_2 e_1 = e_1 \quad e_1 (e_2 e_1) = e_1 e_1 = e_1   \\
        (e_2 e_1)e_1 = e_1 e_1 = e_1 \quad e_2 (e_1 e_1) = e_2 e_1 = e_1   \\
        (e_2 e_2)e_1 = e_2 e_1 = e_1 \quad e_2 (e_2 e_1) = e_2 e_1 = e_1   \\
        (e_2 e_1)e_2 = e_1 e_2 = e_2 \quad e_2 (e_1 e_2) = e_2 e_2 = e_2   \\
        (e_1 e_2)e_2 = e_2 e_2 = e_2 \quad e_1 (e_2 e_2) = e_1 e_2 = e_2
    \end{cases}
\end{align*}
so $e_1,e_2$ are the left identities and the right zeros.
\subsection[10]{Notes}